
% ----------------------------------------------------------
% DADOS DO TRABALHO - CAPA e FOLHA DE ROSTO
% Configure os dados do trabalho aqui
% ----------------------------------------------------------
\titulo{\textbf{Prática da Inovação Social Aberta na Extensão Universitária por meio da concepção de Artefatos Digitais}}
\autor{PEDRO RODOLFO GOMES DE SOUZA}
\local{Recife}
\data{\Year}
\areaconcentracao{\textbf{Área de Concentração}: Sistemas de Informação}
\orientador{\textbf{Orientador}: Kiev Santos da Gama}
%\coorientador{\textbf{Coorientador (a)}: Texto Texto Texto}

\instituicao{UNIVERSIDADE FEDERAL DE PERNAMBUCO \\ CENTRO DE INFORMÁTICA \\PROGRAMA DE PÓS-GRADUAÇÃO PROFISSIONAL EM CIÊNCIA DA COMPUTAÇÃO}
\departamento{Centro de Informática}
\programa{Pós-graduação Profissional em Ciência da Computação}
\emailprograma{prgs@cin.ufpe.br}
\siteprograma{http://cin.ufpe.br/\textasciitilde posgraduacao}

\tipotrabalho{Dissertação de Mestrado}
% O preambulo deve conter o tipo do trabalho, o objetivo, 
% o nome da instituição e a área de concentração 
%\preambulo{Trabalho apresentado ao Programa de Pós-graduação em Ciência da Computação do Centro de Informática da Universidade Federal de Pernambuco, como requisito parcial para obtenção do grau de Mestre Profissional em Ciência da Computação.}

%\preambuloatadefesa{Dissertação apresentada ao Programa de Pós-Graduação Profissional em Ciência da Computação da Universidade Federal de Pernambuco, como requisito parcial para a obtenção do título de Mestre Profissional em 04 de setembro de 2020.}

\preambulo{Dissertação apresentada no Programa de Pós-Graduação Profissional em Ciência da Computação da Universidade Federal de Pernambuco, como um requisito parcial para obter o Título de Mestre em Ciência da Computação.}

%\preambuloatadefesa{Texto texto texto texto texto texto texto texto texto texto texto texto texto texto texto texto texto texto texto texto texto texto texto texto texto texto texto texto texto texto texto texto.}




\input{userlists}



