
\begin{longtable}{|>{\raggedright\arraybackslash}p{5cm}|>{\raggedright\arraybackslash}p{10cm}|}
\caption{Códigos utilizados}
\\
\hline
\rowcolor[HTML]{9B9B9B} 
\textbf{Código} & \textbf{Definição simplificada} \\
\hline
\endfirsthead

\hline
\rowcolor[HTML]{9B9B9B} 
\textbf{Código} & \textbf{Definição simplificada} \\
\hline
\endhead

\hline \multicolumn{2}{r}{{Continua na próxima página}} \\
\hline
\endfoot

\hline
\endlastfoot

Acertos metodológicos & Aspectos positivos da condução metodológica do projeto \\ \hline
Aspirações do futuro & Expectativas futuras relacionadas à carreira \\ \hline
Crescimento profissional & Evolução nas competências exigidas pelo mercado \\ \hline
Desorganização interna & Falta de organização do grupo no desenvolvimento do projeto \\ \hline
Dificuldade na comunicação & Problemas na troca de informações entre os envolvidos do projeto \\ \hline
Dificuldades com ONGs & Problemas relacionados à interação com as ONGs envolvidas \\ \hline
Dificuldades metodológicas & Aspectos negativos da condução metodológica do projeto \\ \hline
Experiência prévia & Contribuições de vivências anteriores similares \\ \hline
Gestão de tempo & Desafios em conciliar o projeto com outras atividades acadêmicas \\ \hline
Melhorias na comunicação Universidade-Prefeitura & Sugestões para uma relação mais eficiente entre Universidade e Prefeitura \\ \hline
Problemas de engajamento & Falta de participação ou motivação nas equipes \\ \hline
Problemas de escopo & Solicitações de escopo mal definidas, sem foco ou sem prioridade \\ \hline
Qualidade das soluções técnicas & Nível de adequação e qualidade das soluções propostas \\ \hline
Relação dialógica & Troca de saberes entre a prefeitura e a universidade \\ \hline
Relação Universidade e Prefeitura & Parceria institucional e sua dinâmica \\ \hline
Retorno à sociedade & Percepção sobre o impacto social gerado pelo projeto \\ \hline
\textit{Soft-skills} & Habilidades interpessoais desenvolvidas no projeto \\ \hline
Sugestões de melhorias metodológicas & Propostas para melhorar a condução do projeto de extensão \\ \hline
\textit{Pivotagem} & Mudanças necessárias durante a execução do projeto \\ \hline
Visão sobre Extensão Universitária & Concepções dos participantes sobre o papel da extensão \\ \hline
Visão sobre Inovação Social & Concepções dos participantes sobre o papel da inovação social \\ \hline
\end{longtable}
{\centering Fonte: O autor (2025). \par}

