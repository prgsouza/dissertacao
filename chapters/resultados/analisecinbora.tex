\subsection{CInbora Impactar}
\label{analisecinbora}

\begin{enumerate}
    \item \textbf{Pré-análise: }
    Foi realizada uma leitura flutuante do material onde foi possível constatar algumas afirmações iniciais que se repetem ao longo das entrevistas e formulários, e serão posteriormente analisadas com mais cuidado, como, por exemplo, uma repetida afirmação da necessidade de monitores ao longo da disciplina.

    Fontes de dados:
    \begin{itemize}
        \item 15 respostas de formulários respondidos por estudantes (Anônimo);
\item 5 entrevistas gravadas com estudantes (Anonimizadas);
\item 4 entrevistas gravadas com colaboradores da Prefeitura (Anonimizadas).
    \end{itemize}


\textbf{Objetivo}:
O objetivo dessa análise é observar quais são os principais acertos e desafios vivenciados pelos estudantes e pelos colaboradores da Prefeitura do Recife no projeto Bora Impactar, visando colher dados para constar em um guia de boas práticas para a prática da Inovação Social Aberta para o Terceiro Setor através da Extensão Universitária.

\textbf{Instrumentos de análise}:
Codificação de segmentos das entrevistas e dos formulários
Exploração dos dados codificados  


 \item \textbf{Exploração do material: }
Códigos:
Foram concebidos 21 códigos ao longo da análise de todo o material: 

\begin{longtable}{|>{\raggedright\arraybackslash}p{5cm}|>{\raggedright\arraybackslash}p{10cm}|}
\caption{Códigos utilizados}
\\
\hline
\rowcolor[HTML]{9B9B9B} 
\textbf{Código} & \textbf{Definição simplificada} \\
\hline
\endfirsthead

\hline
\rowcolor[HTML]{9B9B9B} 
\textbf{Código} & \textbf{Definição simplificada} \\
\hline
\endhead

\hline \multicolumn{2}{r}{{Continua na próxima página}} \\
\hline
\endfoot

\hline
\endlastfoot

Acertos metodológicos & Aspectos positivos da condução metodológica do projeto \\ \hline
Aspirações do futuro & Expectativas futuras relacionadas à carreira \\ \hline
Crescimento profissional & Evolução nas competências exigidas pelo mercado \\ \hline
Desorganização interna & Falta de organização do grupo no desenvolvimento do projeto \\ \hline
Dificuldade na comunicação & Problemas na troca de informações entre os envolvidos do projeto \\ \hline
Dificuldades com ONGs & Problemas relacionados à interação com as ONGs envolvidas \\ \hline
Dificuldades metodológicas & Aspectos negativos da condução metodológica do projeto \\ \hline
Experiência prévia & Contribuições de vivências anteriores similares \\ \hline
Gestão de tempo & Desafios em conciliar o projeto com outras atividades acadêmicas \\ \hline
Melhorias na comunicação Universidade-Prefeitura & Sugestões para uma relação mais eficiente entre Universidade e Prefeitura \\ \hline
Problemas de engajamento & Falta de participação ou motivação nas equipes \\ \hline
Problemas de escopo & Solicitações de escopo mal definidas, sem foco ou sem prioridade \\ \hline
Qualidade das soluções técnicas & Nível de adequação e qualidade das soluções propostas \\ \hline
Relação dialógica & Troca de saberes entre a prefeitura e a universidade \\ \hline
Relação Universidade e Prefeitura & Parceria institucional e sua dinâmica \\ \hline
Retorno à sociedade & Percepção sobre o impacto social gerado pelo projeto \\ \hline
\textit{Soft-skills} & Habilidades interpessoais desenvolvidas no projeto \\ \hline
Sugestões de melhorias metodológicas & Propostas para melhorar a condução do projeto de extensão \\ \hline
\textit{Pivotagem} & Mudanças necessárias durante a execução do projeto \\ \hline
Visão sobre Extensão Universitária & Concepções dos participantes sobre o papel da extensão \\ \hline
Visão sobre Inovação Social & Concepções dos participantes sobre o papel da inovação social \\ \hline
\end{longtable}
{\centering Fonte: O autor (2025). \par}



Dos 21 códigos, houve 205 menções a eles, dos quais, os mais frequentes tratam de:

Acertos metodológicos (19), crescimento profissional (14), \textit{Soft-skills} (13), Relação dialógica (Troca de saberes) (12) e Visão sobre Extensão Universitária (11).

\begin{table}[h]
\centering
\caption{Comparativo por fontes de dados}
\resizebox{\linewidth}{!}{%
\begin{tabular}{|
>{\columncolor[HTML]{9B9B9B}}l |c|c|c|}
\hline
\textbf{Código} & \multicolumn{1}{l|}{\cellcolor[HTML]{9B9B9B}\textbf{Estudantes (Forms)}} & 
\multicolumn{1}{l|}{\cellcolor[HTML]{9B9B9B}\textbf{Estudantes (Entrev.)}} & 
\multicolumn{1}{l|}{\cellcolor[HTML]{9B9B9B}\textbf{Prefeitura}} \\ \hline
\textbf{Acertos metodológicos}    & 6 & 12 & 1 \\ \hline
\textbf{Soft-skills}              & 11 & 2  & 0 \\ \hline
\textbf{Crescimento profissional} & 6  & 8  & 0 \\ \hline
\textbf{Problemas de escopo}      & 1  & 3  & 3 \\ \hline
\textbf{Relação dialógica}        & 4  & 6  & 2 \\ \hline
\end{tabular}}
\vspace{0.5em}
{Fonte: O autor (2025).}
\end{table}


Essa distribuição demonstra que os estudantes enfatizaram o aprendizado e a prática, enquanto os colaboradores da prefeitura apontam a questão operacional.

É possível também inferir que boa parte do que foi trazido pelos estudantes e pelos colaboradores da prefeitura falam da forma da condução em si do projeto, e sobre a importância da troca de saberes que houve entre a universidade e a prefeitura, através da Inovação Social Aberta.

    \item \textbf{Tratamento dos resultados e interpretação: }
Principais acertos:
Relação dialógica: Os estudantes em geral afirmaram que a metodologia de um projeto de extensão em parceria com a Prefeitura auxiliou tanto na absorção dos conteúdos ministrados na disciplina, quanto no processo de aprendizagem autêntica, onde muitos nunca haviam tido anteriormente a oportunidade de participar de um projeto que seria executado de fato, e não somente um projeto simulacro para uma disciplina, vivenciando uma relação dialógica, onde ambos os lados tiveram um ganho;

Desenvolvimento pessoal: Muitos estudantes relataram que desenvolveram muitas habilidades enquanto vivenciavam o desafio de um projeto que estava sendo executado em tempo real, e que iria impactar diversas pessoas, por trabalhar com o terceiro setor. Segundo os estudantes, as três \textit{soft-skills} que mais foram desenvolvidas foram as seguintes: Trabalho em equipe (100\%), habilidades organizacionais (94,1\%) e resolução de problemas (94,1\%);

Comunicação mais direta: Os estudantes e os colaboradores da prefeitura elogiaram a comunicação mais direta entre todas as partes, e os estudantes reforçaram a interação com o professor por meio de ferramentas digitais de comunicação, ressaltando principalmente o atendimento de demandas fora do horário das aulas.

Esses acertos são verificáveis através das seguintes afirmações:

“Quando comecei a cadeira, eu só sabia Python.Agora, tipo, sei até mexer em Figma, em design, essas coisas. Então, tipo, tá ótimo.Tô quase \textit{full-stack}, então foi muito proveitoso.”
(Entrevista — Estudante 01)

“Estar em um cenário real, porque eu pude ter contato com pessoas reais, com projetos reais que realmente pessoas precisam deles e também com stakeholders reais que tinham seus pedidos”
(Entrevista — Estudante 02)

“Gosto de cadeiras que colocam, por exemplo, essa do professor, do período passado. Achei bem legal porque pode colocar realmente conteúdo que a gente aprende de forma prática, assim, o jeito que eu mais aprendo, sabe?”
(Entrevista — Estudante 03)

“[Aprendemos]As variáveis mais críticas em um ambiente de desenvolvimento. Em um ambiente de projetos e tal. Como seria esse tipo de simulação em uma empresa grande. Qual é o tipo de projeto de desenvolvimento? Ensinar a base da engenharia de software.”
(Entrevista — Estudante 04)

“A seriedade da coisa me fez. Ah! Agora eu não sou a pessoa que fazia um projeto em HTML e CSS e mandava no \textit{Classroom}. Não, é um projeto real, fixo com a prefeitura. Então, essa seriedade gostei muito.”
(Entrevista — Estudante 05)

“Porque, querendo ou não, cada um vai dominar algo. E, quando a gente junta isso, a gente junta o melhor de cada um, né? Então, essa parte da gente trazer conhecimento técnico, conhecimento ali que está sendo formado, né?”
(Entrevista — Colaborador da Prefeitura 01)

“Os meninos não fizeram uma coisa assim, um arroz com feijão, não. Fizeram uma coisa linda, bonita. Eu me emocionei, chorei umas três vezes com eles apresentando.”
(Entrevista — Colaborador da Prefeitura 03)

Principais dificuldades:
Monitoria: Um ponto muito colocado pelos estudantes foi de que a presença somente do professor em sala de aula foi um gargalo para uma execução mais efetiva de alguns pontos do projeto, pois, em decorrência do quantitativo de estudantes participantes do projeto, o professor não conseguia dar uma atenção plena a todas as equipes quando necessário;

Problemas de escopo: Algumas equipes acreditam que tiveram seu desempenho no projeto um pouco prejudicado por conta da falta de um escopo mais acertado por parte da prefeitura, pois ao longo da execução, perceberam que o tema que estavam trabalhando não eram prioridades, mesmo tendo sido anunciados pela prefeitura. Algumas equipes também relataram que a prefeitura deveria delimitar melhor os projetos, para maior efetividade das soluções apresentadas, com menor retrabalho, e assim, com menor de tempo de desenvolvimento;

Dificuldade de execução durante as disciplinas: alguns estudantes apontaram que a execução do projeto ocorreu em um período que existiam disciplinas muito pesadas, e que exigiam uma atenção maior, e muitos aproveitaram o período de férias para poderem se dedicar mais ao projeto de extensão.

Essas dificuldades são verificáveis através das seguintes afirmações:

“A gente precisava mais de um guia técnico. Tipo, tem que ir com a tecnologia. A gente precisa aprender essas coisas.Tipo, aquilo foi um mar totalmente obscuro, que a gente foi desbravando.”
(Entrevista — Estudante 01)

“Com a \textit{stakeholder}, a responsável da prefeitura, às vezes, era um pouco confusa sobre o que ela queria. Ela ia descobrindo o que ela queria ao longo do projeto.E aí, a gente tinha que fazer algumas mudanças no projeto.Mas, no geral, foi uma ótima relação com ela.”
(Entrevista — Estudante 02)

“Seria bom um plano de aula desde o início, já com as entregas que precisam ser feitas. Para os alunos terem um direcionamento melhor. Um planejamento de entregas.”
(Entrevista — Estudante 02)

“Essa dos monitores é uma [melhoria], que ajudam bastante. Indicação de \textit{stack}. Por exemplo, [...] início da cadeira a gente ficou muito sem saber qual biblioteca se encaixava com qual [...] banco.”
(Entrevista — Estudante 03)



“Talvez para não conseguir achar palavras, a gente precisa criar uma coisa mais duradoura que esse tipo de estratégia em parceria continue independente de quem esteja.”
(Entrevista — Colaborador da Prefeitura 02)

“Houve em alguns momentos uma certa demora no envio de informações e ajustes com o nosso desenvolvedor.”
(Entrevista — Colaborador da Prefeitura 03)

“Eu acredito que é o risco que você toma ao criar uma relação com estudantes em uma cadeira que tem um tempo muito curto.”
(Entrevista — Colaborador da Prefeitura 04)

“Quanto mais rápido a gente tiver esse escopo bem definido, mais rápido a gente parte para a implementação.”
(Entrevista — Colaborador da Prefeitura 04)


\item \textbf{Lições aprendidas:}
\begin{itemize}
    \item \textbf{Maior necessidade de padronização no projeto:} criação de guias técnicos, cronogramas claros, e ferramentas a serem utilizadas por todas as equipes;
    \item \textbf{Acompanhamento mais próximo:} por meio de monitores e reuniões mais frequentes;
    \item \textbf{Sistematização e validação direta com o parceiro:} estabelecimento de ponto de contato fixo, cronograma de \textit{feedback} com o cliente, além da revisão dos requisitos e temas propostos para melhor assertividade.
\end{itemize}

\end{enumerate}

\textbf{Conclusão da análise do CInbora Impactar}

A análise de conteúdo realizada nas entrevistas do projeto Bora Impactar mostraram que o mesmo foi vivenciado pelos estudantes como uma prática de aprendizagem ativa, onde os estudantes são colocados no centro do processo de aprendizagem, encorajados a participar de forma ativa e colaborativa na construção do conhecimento, onde nesse caso, foi no desenvolvimento das soluções tecnológicas solicitadas pela prefeitura. A prefeitura também logrou momentos de aprendizagem com os estudantes acerca de melhorias sugeridas pelos mesmos na forma de condução do projeto Bora Impactar para além das fronteiras da universidade. Mesmo com diversos desafios apresentados ao longo da execução, foi notório o benefício da prática da Inovação Social Aberta por meio da Extensão Universitária para o terceiro setor.
