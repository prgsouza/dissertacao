\section*{\#R7 – Definir Configurações de Extensão: Integradas a Disciplinas ou Autônomas}

Com o advento da curricularização da extensão nas universidades, cada vez mais têm se oferecido disciplinas já integradas com carga horária de extensão universitária, porém, cada projeto de extensão terá sua própria dinâmica, que poderá se adequar mais a um tipo específico de configuração de extensão.

Os projetos de extensão podem ocorrer integrados a uma disciplina ou de maneira autônoma.

\textbf{Projetos integrados a disciplinas (ex.: CInbora Impactar):}
\begin{itemize}
    \item Conteúdo teórico aplicado simultaneamente.
    \item Projetos reais em vez de simulações.
    \item Maior engajamento dos estudantes na disciplina e/ou projeto.
\end{itemize}

\textbf{Pontos negativos:}
\begin{itemize}
    \item Pode competir com outras disciplinas do mesmo semestre.
    \item Um excesso de equipes pode dificultar o gerenciamento.
    \item Participação pode se tornar impessoal devido à quantidade de estudantes.
\end{itemize}

\textbf{Projetos autônomos (ex.: CInovação Social):}
\begin{itemize}
    \item Equipes menores podem gerar participação mais pessoal e empática.
    \item Tempo de duração variável, podendo incluir períodos de férias.
\end{itemize}

\textbf{Pontos negativos:}
\begin{itemize}
    \item Possível desmotivação por ausência de impacto acadêmico perceptível a curto prazo.
    \item Dificuldade no financiamento das atividades.
\end{itemize}
