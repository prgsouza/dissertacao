\section*{\#R10 – Empregar Técnicas de Ideação e Geração de Soluções}

Em alguns casos, os estudantes já podem possuir experiência prévia em ideação e geração de soluções, trazendo ideias de disciplinas e \textit{ideathons}/\textit{hackathons} que já participaram anteriormente. Porém, as instituições podem não possuir a mesma familiaridade com algumas técnicas mais conhecidas. É extremamente importante que uma boa explanação e nivelamento para que todos os envolvidos possam participar ativamente, e sem receio do processo de cocriação.

Algumas metodologias que foram experimentadas nos projetos analisados:
\begin{itemize}
    \item \textbf{Mapa de empatia}: exercício dos estudantes se colocarem no lugar das pessoas atendidas pela ONG se sentem, o que elas veem e ouvem, o que falam e fazem. Isso ajuda a entender o que realmente importa para elas.
    \item \textbf{Como podemos}: essa etapa irá transformar os problemas encontrados em perguntas que começam com “Como podemos...?”, para abrir espaço para ideias criativas e soluções práticas;
    \item \textbf{\textit{Brainstorm}}: essa etapa será um estímulo para todos pensarem em muitas ideias diferentes, sem julgar. Quanto mais ideias, melhor, e depois a triagem das que são mais úteis e fáceis de fazer.
    \item \textbf{SCAMPER}: é o acrônimo de Substituitr, combinar, adaptar, modificar, propor, eliminar e reorganizar, as quais são fases que visam desenvolvimento da criatividade, analisando todos os aspectos de um produto, seu público alvo e seu mercado;
    \item \textbf{Análise de concorrentes}: mapeamento de soluções já existentes no mercado para resolução do problema, e suas vantagens e desvantagens;
    \item \textbf{Seleção de ideias}: visando considerar todas as ideias envolvidas,  votar nas ideias que têm mais impacto para a ONG e que sejam factíveis de serem feitas no tempo do projeto, assim escolhendo as que serão trabalhadas;
    \item \textbf{Histórias de Usuário}: escrever pequenas frases acerca das ideias escolhidas, que mostram quem vai usar, o que quer fazer e qual benefício espera. Isso ajuda a organizar o trabalho.
\end{itemize}

É importante também que seja realizada uma prototipagem do que será construído, visando auxiliar no processo de co-criação e principalmente do atendimento dos requisitos levantados. Também traz uma maior motivação aos estudantes e aos outros \textit{stakeholders}, ao começarem a visualizar o trabalho de fato, sendo concretizado, e podem ser feitos das seguintes formas:

\begin{itemize}
    \item \textbf{\textit{Wireframe}}: dar uma "cara" a tudo o que foi construído ao longo do processo do Ideathon, através de desenhos em papel do que foi imaginado para a interface do artefato;
    
    \item \textbf{Figma}: protótipos de maior fidelidade comparados aos anteriores, e auxiliam para melhor entendimento de como a plataforma deverá, de fato, se comportar.

\end{itemize}

Os protótipos devem ser validados com os envolvidos antes da implementação. Por conta disto, é importante que os projetos sejam demonstrados e explicados a instituição parceira, para que a mesma mostre sua opinião acerca do que está sendo construído, além de também poderem aprender com o processo que está sendo realizado para essa solução.

