\section*{\#R11 – Desenvolver Artefatos Digitais adequadamente}

Segundo o preconizado pela metodologia \textit{Speedplay} de \citeauthor{ferrario2014} (\citeyear{ferrario2014}), os artefatos digitais podem ser utilizados como meio de transformação social, o qual é o foco deste trabalho. 

Existem diversas formas de construção de artefatos digitais em projetos de extensão, e as formas vivenciadas e acertadas, que ocorreram nos projetos do Centro de Informática avaliados por essas recomendações, são as seguintes:

\begin{itemize}
    \item Programação manual: os estudantes desenvolvem todos os códigos e interfaces dos artefatos digitais. Recomendado para equipes grandes e períodos de execução mais longos.
    \item Programação via Inteligência Artificial (\textit{vibecoding}): os estudantes desenvolvem códigos e interfaces por meio da inteligência artificial, realizando manualmente correções e adaptações necessárias. Recomendado para equipes menores e períodos de execução mais curtos, permitindo maior foco nas ideias, processos e entendimento da dor da instituição.
\end{itemize}

Como a forma da realização da programação manual pode se dar de diversas formas, utilizando as mais diversas metodologias e \textit{frameworks} que já estão consolidados e são bastante conhecidos pela literatura científica, este trabalho vai trazer uma nova perspectiva, que está em amplo crescimento, e trazer algumas recomendações básicas de como utilizar exitosamente, o qual é o \textit{vibecoding}.

A forma de programação por meio de Inteligência Artificial tem se mostrado em evidência, em decorrência do crescimento súbito do tema no mercado e na mídia, o que também pode ser um motivador maior para os estudantes, por terem a oportunidade de trabalharem com novas tecnologias que estão cada vez mais em evidência no cotidiano do mercado. 

Ferramentas úteis no Vibecoding:
\begin{itemize}
    \item \textbf{ChatGPT}: visando uma melhor assertividade da utilização dos créditos gratuitos disponibilizados pelo V0/Vercel, os estudantes utilizaram o ChatGPT para melhor criação do \textit{prompt} conforme os requisitos levantados em conjunto com a ONG e com a interface que foi prototipada, seja no Wireframe ou no FIGMA, onde o ChatGPT permite a adição de fotos, facilitando na construção do \textit{prompt};

    Exemplo de \textit{prompt} utilizado no projeto CInovação Social: Crie um prompt para o v0 com base nesse projeto (descrição básica do projeto) e nesse protótipo (enviado imagem do protótipo do FIGMA).
    
    \item \textbf{V0 - Vercel}: as interfaces (front-end) do projeto CInovação Social e seu banco de dados foi todo gerado via IA através do V0, onde os estudantes, através do \textit{prompt} gerado pelo ChatGPT, gerou o código em React/Next.js com Tailwind, entregando componentes reutilizáveis, e também gerando o \textit{back-end} e o banco de dados completo no Supabase, além de realizar automaticamente a integração destes 3 componentes da plataforma. Em decorrência do V0 ser pago e possuir poucos créditos gratuitos, foi utilizado para gerar o núcleo duro do artefato, onde os estudantes poderão trabalhar em cima do projeto, refinando o mesmo. O tempo economizado durante a construção dos principais componentes necessários para a solução pode ser utilizado para a documentação e melhoria incremental do projeto.
    
    \item \textbf{Gemini}: diversos erros foram ocorrendo durante o processo de criação da solução através do V0/Vercel, e para a correção dos erros, seja de interface ou do funcionamento na lógica, cada correção era cobrada a parte pela plataforma, e muitas das vezes as soluções eram incompletas, gerando mais cobranças para novas correções. Em vez de realizar as correções pelo V0/Vercel, os erros e os códigos foram enviados para o Gemini, onde o mesmo realizava a proposta de correção, que era analisada pelos estudantes e era integrada de fato no projeto. Alguns erros se repetiam em diferentes módulos do sistema, e ao corrigir algum caso específico junto ao Gemini, os estudantes conseguiram otimizar o tempo aplicando as mesmas correções em outros módulos, gerando também um aprendizado coletivo sobre padrões de erro e correções no código.
    
    \item \textbf{Copilot}: essa ferramenta foi utilizada durante o processo de codificação por parte dos estudantes, onde a correção de erros do V0/Vercel e do Gemini não foi efetiva, e precisava de um olhar mais detalhado pelos estudantes, através da utilização de ferramentas de codificação comuns como o Visual Studio Code, em auxilio do Copilot para correções mais pontuais via IA.
\end{itemize}
