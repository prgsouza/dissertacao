\section*{\#R4 – Agir com respeito e empatia com o espaço e tempo da instituição parceira}

A extensão universitária exige um esforço de empatia por parte dos estudantes, para compreenderem os reais desafios que as instituições do terceiro setor vivenciam. Elas não são empresas que possuem horários rígidos, cronogramas inflexíveis, mas sim um organismo vivo que possui inúmeros desafios diários, além de vivenciarem um quantitativo extremamente reduzido de recursos humanos e recursos financeiros, como apontado por \citeauthor{gama2023} (\citeyear{gama2023}). O respeito e empatia deste momento é extremamente importante para o fortalecer o processo da relação dialógica.

No projeto CInovação social, ocorreu uma situação na qual a representante da instituição necessitou remarcar um dos encontros para participar de um evento para captação de recursos para a própria instituição. 

Outra situação foi de algumas interrupções que ocorreram no \textit{Status Report}, por parte da representante da instituição, que necessitou de dar uma rápida atenção para seu filho pequeno, ao mesmo tempo que participava da reunião e se desdobrava para resolver outras questões. 

Essas situações exigem um nível de empatia e compreensão dos estudantes, e da equipe executora do projeto e também são experiências valiosas, por colocarem os estudantes confrontando a realidade vivenciada pelas instituições do terceiro setor, onde muitos dos seus colaboradores acabam executando diversas tarefas nos mais diversos escopos de atuação, por conta da necessidade e da falta de voluntários para os auxiliar.

Quem está liderando o projeto deve estimular a empatia e o respeito por parte dos estudantes e de todo o grupo, e sempre ressaltar que essas situações não devem ser entendidas como uma falta de compromisso por parte da instituição, mas sim uma parte pulsante e viva da instituição.
