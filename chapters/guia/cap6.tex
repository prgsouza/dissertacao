\section*{\#R6 – Valorizar Processos de Aprendizagem e Engajamento e Desenvolvimento de Soft-Skills}

O processo de ensino-aprendizagem é uma das partes mais importantes da extensão. Muitas das vezes se existe um grande foco na solução que será entregue, mas esta solução será somente um reflexo de todo o processo de ensino-aprendizagem vivenciado tanto pelos estudantes quanto pela instituição parceira no processo de cocriação e de inovação aberta.

O processo de inovação social aberta ocorre através da troca de aprendizado e da relação dialógica:
\begin{itemize}
    \item A instituição aprende com os estudantes.
    \item Os estudantes integram teoria e prática em experiências reais e significativas.
\end{itemize}

Isso potencializa o desenvolvimento de \textit{soft-skills}, como resolução de problemas, gerenciamento de times e pensamento crítico. É importante resgatar o protagonismo estudantil, tornando os projetos autogerenciáveis. O coordenador, entretanto, deve acompanhar constantemente, garantindo que o projeto funcione e os estudantes não se sintam somente força de trabalho.

Ao usar metodologias ágeis, funções (ex.: \textit{Product Owner}, \textit{Scrum Master}) devem ser definidas desde o início, com apoio do coordenador no desenvolvimento de habilidades gerenciais. O coordenador também pode atuar diretamente na execução, incluindo a confecção do artefato digital, gerando maior engajamento e segurança nos estudantes.

O registro de todas as atividades e decisões é fundamental para auxiliar na execução do projeto e para lições aprendidas em projetos futuros.
