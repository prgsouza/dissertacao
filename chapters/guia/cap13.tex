\section*{\#R13 – Avaliar as soluções propostas e o aproveitamento dos estudantes}

A avaliação em projetos de extensão pode ser algo complexo, pois não é um processo de mensuração tal qual uma prova de alguma disciplina. O processo de extensão toca em diversos pontos que não são tangíveis e não são avaliados por um simples barema. O processo avaliativo não deve jamais levar somente em conta a qualidade dos produtos finais, mas sim todo o processo de ensino-aprendizagem vivenciado pelos estudantes e pela instituição e a relação dialógica que se estabeleceu ao longo das vivências do projeto.

A avaliação deve ser pensada de uma forma multidimensional, e deve envolver todos os participantes do projeto, tanto os que fazem parte da universidade, quanto os que fazem parte da instituição do terceiro setor parceira e, se possível e aplicável, da comunidade ao redor.

Formas de avaliação:

\begin{itemize}
    \item \textbf{Avaliação 360:} esse tipo de avaliação permite que todos os participantes avaliem todas as partes envolvidas, tanto da universidade quanto da instituição, e por ser anônima, permite uma avaliação menos enviesada e mais sincera;
    \item \textbf{Avaliação por pares:} os estudantes podem se avaliar mutualmente ao longo da execução do projeto, estimulando a responsabilidade mútua;
    \item \textbf{Autoavaliação reflexiva:} os estudantes podem avaliar a si e atribuírem notas para cada critério, levando-os a momentos de autocrítica e de desenvolvimento de responsabilidade;
    \item \textbf{Avaliação pela instituição e/ou comunidade:} esse momento é o de verificar se as soluções propostas de fato auxiliam a resolver as problemáticas colocadas ao longo da execução do projeto, e se os principais requisitos foram atendidos.
\end{itemize}

É importante que as formas de avaliação sejam utilizadas combinadamente, auxiliando assim na efetividade do projeto. Também deve ser ressaltado para os estudantes que o processo avaliativo serve para o aprimoramento do processo extensionista, e maior assertividade em próximas execuções.