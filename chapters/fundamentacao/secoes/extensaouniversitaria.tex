\section {Extensão Universitária}
\label{extensaouniversitaria}

As universidades possuem três missões: ensino, pesquisa e a terceira missão, que envolve a relação entre a universidade e a sociedade. \cite{klaumann2023}

Segundo Compagnucci e Spigarelli (2020) apud \citeauthor{correa2021} (\citeyear{correa2021}), “A TM [terceira missão] é soma de todas as atividades relacionadas com a geração, uso, aplicação e exploração do conhecimento universitário, capacidades e recursos, fora do ambiente acadêmico”.

Essa terceira missão pode ser realizada por meio de atividades de extensão no âmbito das universidades, contribuindo assim para a geração de inovações sociais. \cite{klaumann2023}

“A Extensão Universitária é o processo educativo, cultural 
e científico que articula o Ensino e a Pesquisa de forma 
indissociável e viabiliza a relação transformadora entre 
Universidade e Sociedade.” (FORPROEX, 1987, apud \citeauthor{forproex2016}, \citeyear{forproex2016}, p. 29).

A Extensão Universitária é muito mais do que uma mera formalidade educacional para cumprir metas ou obrigações curriculares. É um legítimo processo de socialização do conhecimento da academia, para uma intervenção no meio em que está inserida, auxiliando na transformação daquela realidade, em uma coletividade entre a universidade e a sociedade. É a concepção de que a universidade volta-se para os problemas sociais, e que irá auxiliar na resolução destes por meio de todo conhecimento produzido dentro dela.

Segundo o \citeauthor{forproex2016} (\citeyear{forproex2016}), a Extensão Universitária é trabalhada por meio das seguintes áreas temáticas:

\begin{quadro}[H]
\centering
\caption{Áreas temáticas da Extensão Universitária}
\begin{tabular}{|c|c|}
\hline
\textbf{Comunicação}                & \textbf{Meio-Ambiente}         \\ \hline
\textbf{Cultura}                    & \textbf{Saúde}                 \\ \hline
\textbf{Direitos Humanos e Justiça} & \textbf{Tecnologia e Produção} \\ \hline
\textbf{Educação}                   & \textbf{Trabalho}              \\ \hline
\end{tabular}
\newline
\newline
Fonte: O autor, adaptado de \cite{forproex2016}.
\end{quadro}


As diretrizes que devem nortear as ações extensionistas das universidades devem ser as seguintes (Nogueira, 2000, apud \citeauthor{forproex2016}, \citeyear{forproex2016}):

\begin{itemize}
    \item \textbf{Interação dialógica}: a relação entre a universidade e a sociedade deve ser uma via de mão dupla, um diálogo e troca de saberes, e não somente uma imposição da hegemonia acadêmica sobre a sociedade, ou de uma extensão dos saberes acadêmicos, mas sim uma construção coletiva, onde ambos os lados contribuem na construção do conhecimento, reconhecendo assim a importância dos conhecimentos advindos da sociedade.
    \item \textbf{Interdisciplinaridade e Interprofissionalidade}: ao atuar em problemas da sociedade, que possuem naturezas complexas, é necessária a atuação de várias áreas do conhecimento, tanto disciplinar quanto profissional. A extensão universitária proporciona uma visão holística por meio de práticas que combinam a especialização e consideram a complexidade das necessidades sociais, resultando em uma maior assertividade das práticas realizadas.
    \item \textbf{Indissociabilidade Ensino-Pesquisa-Extensão}: o tripé universitário é completamente essencial para a existência por si só de cada um de seus componentes, que são inseparáveis. O ensino proporciona todo o cabedal acadêmico de conhecimento necessário para que o estudante, como protagonista do processo, o coloque em prática na sociedade por meio da extensão universitária, e também para a geração de novos conhecimentos por meio da pesquisa.
    \item \textbf{Impacto na formação do estudante}: a extensão tem um importante papel na vida acadêmica dos estudantes por proporcionar experiências e conhecimentos que muitas das vezes não são desenvolvidos em sala de aula, como, por exemplo, \textit{softskills} (Liderança, coletividade, comunicação, e outras) além da vivência entre o conhecimento acadêmico e a realidade social.
    \item \textbf{Impacto e transformação social}: essa diretriz coloca a Extensão Universitária em um caráter mais politizado, onde a coloca como o mecanismo que conecta a universidade com a sociedade, possibilitando a transformação não somente da sociedade, mas também da própria universidade.
\end{itemize}

Segundo a Resolução n.º 7 do \gls{CNE}/\gls{MEC} (\citeyear{cne2018}), a Extensão Universitária deve compor no mínimo 10\% da carga horária dos cursos de graduação no Brasil, e como determina a Resolução sobre Curricularização da Extensão no âmbito da \gls{UFPE} (\citeyear{ufpe2022}), essas ações devem ser orientadas prioritariamente para as áreas de grande pertinência social. As atividades caracterizadas como Atividades Extensionistas são as seguintes \cite{proexuepb}:
\begin{itemize}
    \item \textbf{Programas:} um conjunto de ações extensionistas, que estejam associados com o ensino e pesquisa, e preferencialmente trabalhado transdisciplinarmente, envolvendo diversas áreas do conhecimento na universidade. Possui um prazo mais longo em comparação as outras atividades.
    \item \textbf{Projetos:} um conjunto de ações continuas, com um objetivo definido e prazo específico.
    \item \textbf{Cursos:} um conjunto de atividades com carga teórica e/ou prática, com métodos avaliativos formais, carga horária mínima, e com a finalidade da disseminação do conhecimento.
    \item \textbf{Eventos:} ação, apresentação ou exibição pública, de disseminação de conhecimento, cultura, esportes, por meio de debates, seminários, congressos, simpósios e afins.
    \item \textbf{Prestação de Serviços:} a execução ou participação em serviços realizada pela universidade à comunidade, empresas, órgãos, dentre outros.
    \item \textbf{Produção e publicação:} as publicações e produtos realizados através das atividades desenvolvidas nas ações de extensão.
\end{itemize}