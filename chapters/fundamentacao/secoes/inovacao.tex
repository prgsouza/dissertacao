\section{INOVAÇÃO}
\label{inovacao}

A Inovação é um amplo conceito, abrangendo diversas áreas do conhecimento, ganhando cada vez mais destaque na academia e no mercado. A inovação voltada para negócios, é um produto ou processo empresarial novo ou melhorado (ou a combinação de ambos), que difere dos produtos ou processos anteriores da organização, sendo introduzido no mercado por ela. \cite{ocde2018}

O termo foi popularizado pelo economista austríaco Joseph Schumpeter, em seu livro \textit{Teoria do Desenvolvimento Econômico,} publicado pela primeira vez no ano de 1964. Em seu livro, a ideia de \textbf{“Inovação”} é trazida através do termo \textbf{“Novas combinações”}. Ao falar das novas combinações, ele faz um paralelo direto ao sistema de produção. Ele trata a produção como a combinação de materiais e forças que estão ao alcance dos produtores, e a inovação, ou “combinação nova” o motor do desenvolvimento das organizações e industrias, e elenca os seguintes tipos de inovação (\citeauthor{schumpeter1997} \citeyear{schumpeter1997}, p. 76):

\begin{itemize}
    \item \textbf{Introdução de um novo bem:} a introdução de um novo produto ou serviço no mercado. Esse processo de introdução também pode ser um aprimoramento de produtos ou serviços já existentes;
    \item \textbf{Introdução de um novo método de produção:} possui o foco na criação de um novo processo não utilizado anteriormente na organização. Também pode ser a melhoria de algum já existente;
    \item \textbf{Abertura de um novo mercado:} criação de novos mercados ou novo segmento de clientes, a partir da percepção da organização de um nicho de mercado não explorado por seus concorrentes;
    \item \textbf{Novas fontes de oferta:} conquista por parte da empresa de inovação nas fontes de oferta de matérias-primas, bens semimanufaturados ou insumos para abastecimento de sua produção;
    \item \textbf{Estabelecimento de uma nova organização:} criação de novas formas de estrutura e gestão nas corporações, mediante novos métodos de trabalho, estruturas, organizações, e afins.
\end{itemize}

Outra abordagem extremamente importante quando se fala de inovação, são os conceitos trazidos por Clayton Christensen no seu livro \textit{O Dilema da Inovação}, tratando dos impactos trazidos pela inovação, por meio de exemplos reais da indústria, que podem ser classificados como \cite{christensen2012}:

\begin{itemize}
    \item \textbf{Inovação Incremental:} é o tipo de inovação mais comum, que possui uma natureza progressiva, buscando melhorias ou atualizações de produtos, serviços ou processos já existentes numa organização. É a inovação que possui um menor custo para a organização ao ser executada. Pode trazer um retorno de curto prazo, por ser facilmente superada pela concorrência;
    \item \textbf{Inovação Disruptiva:} é a inovação que rompe as barreiras do mercado, com uma natureza mais radical, alterando profundamente todo o setor, ultrapassando os limites atuais. É o tipo de inovação mais custoso para a organização, porém, pode trazer um retorno com maior prazo, por ser a inovação mais complexa de ser superada.
\end{itemize}

Além disso, há a classificação quanto ao modelo de Inovação, que será tratado com mais aprofundamento neste trabalho em capítulo dedicado, conforme a perspectiva de Henry Chesbrough no seu livro \textit{Inovação aberta} publicado pela primeira vez no ano de 2003, pontuando os seguintes modelos de inovação \cite{chesbrough2003}:

\begin{itemize}
    \item \textbf{Inovação fechada:} é realizada pelas organizações de maneira única e exclusivamente interna, na qual não existe nenhuma participação de atores externos, nem da exportação da inovação de uma organização a outra. É um tipo de inovação mais conservadora, utilizando somente os recursos disponíveis na própria organização.
    \item \textbf{Inovação aberta:} é possibilitada a participação de atores externos as organizações, que podem contribuir ativamente no processo inovatório, trazendo pontos de vistas diferentes. Além disso, pode ser a exportação de competências e habilidades de uma organização para outra.
\end{itemize}

Com base nos conceitos demonstrados sobre Inovação, a tabela abaixo oferece um panorama considerando seu tipo, natureza e modelos:

\begin{quadro}[H]
\centering
    \caption{Tipos de Inovação}
    \begin{tabular}{|>{\columncolor[HTML]{C0C0C0}}l|c|l|l|c|l|}
        \hline
        \rowcolor[HTML]{C0C0C0}
        \multicolumn{6}{|c|}{\textbf{Inovação}} \\ \hline
        \textbf{Tipo}     & Produto & Processo & Mercado & Recursos & Organizacional \\ \hline
        \textbf{Natureza} & \multicolumn{3}{c|}{Incremental} & \multicolumn{2}{c|}{Disruptiva} \\ \hline
        \textbf{Modelo}   & \multicolumn{3}{c|}{Fechada} & \multicolumn{2}{c|}{Aberta} \\ \hline
    \end{tabular}
\newline
\newline
Fonte: O autor (2024).
\end{quadro}


