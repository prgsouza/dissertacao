\section{Organização do Trabalho}
\label{organizacao}

Esse trabalho se organizará em seções, através da seguinte organização e estrutura:
\par\vspace{1\baselineskip}

\begin{itemize}
    \item \textbf{Fundamentação Teórica}: todo o arcabouço teórico-científico necessário para uma compreensão inicial sobre os principais temas que permearão esse trabalho: Inovação Aberta, Inovação Social, Inovação Social Aberta e Extensão Universitária, onde todos esses temas serão tratados pela ótica dos principais pesquisadores de cada área;
    \item\textbf{Revisão de Literatura:} protocolo da Revisão Rápida em bases de dados acadêmicas conceituadas na área da Ciência da Computação;
    \item \textbf{Metodologia de Pesquisa}: explanação da metodologia de Estudo de Caso e Pesquisa-Ação, que será adotada para a condução da pesquisa proposta;
    \item \textbf{Resultados }: ocorrerá a apresentação e discussão dos resultados obtidos através da Revisão Rápida da Literatura, consoante o protocolo proposto e executado na seção de Metodologia de Pesquisa deste trabalho, além das análises de dados dos Projetos de Extensão, e a proposta do guia de boas práticas para a inovação social aberta voltada ao terceiro setor através da extensão universitária;
    \item \textbf{Considerações finais}: apresentação das considerações finais desta proposta de dissertação, as contribuições realizadas até o exato momento.

\end{itemize}






