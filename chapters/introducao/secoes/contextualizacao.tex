\section{Contextualização}
\label{contextualizacao}

A Inovação Social, segundo a \gls{OCDE} \cite{ocde2024social} é a área que busca soluções possíveis para as questões sociais, visando a melhoria do bem-estar e qualidade de vida das pessoas e comunidades em geral, através de serviços, produtos, dentre outros. Esse conceito se relaciona diretamente com o conceito de Economia Social, sendo um dos motores desta, para atender as necessidades da sociedade, por meio de conjuntos de organizações sociais (Organizações Não-Governamentais, Cooperativas, Organizações Sem Fins Lucrativos, dentre outras).

A Inovação Aberta é um termo cunhado pelo professor universitário Henry Chesbrough, no seu livro chamado \textit{Inovação Aberta} (2003). Segundo \citeauthor{chesbrough2003}, (\citeyear{chesbrough2003}), a inovação aberta é um novo paradigma inovativo, buscando “inovar a inovação”.
Esse paradigma traz a concepção de que as empresas podem potencializar seus processos inovativos através da abertura a ideias, tecnologias e conhecimentos externos, vindo de outras organizações, \textit{stakeholders} e afins. Essa concepção de modelo de inovação pode ser bastante benéfica nas organizações por reduzir custos com retrabalho, permitindo a observação e o \textit{benchmark} das melhores práticas, baseado na experiência das outras organizações.

A Inovação Aberta não é um conceito utilizado somente no contexto da indústria. Estão surgindo cada vez mais novos usos para a Inovação Aberta, aplicada a Governos, Organizações da Sociedade Civil, e outros. O conceito deste trabalho é o da Inovação Social Aberta, um tipo específico de aplicação da Inovação Aberta para necessidades sociais.

A Inovação Social Aberta é a junção dos dois conceitos apresentados anteriormente, onde, no caso, a Inovação Aberta é utilizada como um meio de potencialização da Inovação Social, agora, não mais visando maximização do lucro da organização, mas sim, a aplicação de abordagens inovadoras para realizar práticas que atendam as necessidades da sociedade e os objetivos de melhoria de bem-estar e qualidade de vida desta. Enquanto os inovadores do mercado e da indústria focam seus resultados nos seus lucros e retornos, os inovadores sociais focam na mudança social como seu principal retorno. \cite{chesbrough2014}. Uma das formas de realização desta Inovação Social Aberta é a Extensão Universitária.

A Câmara de Educação Superior do \gls{CNE} do \gls{MEC}, traz a seguinte definição de Extensão na sua Resolução n.º 7 de 2018: 
\begin{citacao}“A Extensão na Educação Superior Brasileira é a atividade que se integra à matriz curricular e à organização da pesquisa, constituindo-se em processo interdisciplinar, politico educacional, cultural, científico, tecnológico, que promove a interação transformadora entre as instituições de ensino superior e os outros setores da sociedade, por meio da produção e da aplicação do conhecimento, em articulação permanente com o ensino e a pesquisa. (\citeauthor{cne2018} \citeyear{cne2018}, p. 1-2).”
\end{citacao}

Através dessa definição, é possível observar que a Extensão tem como um importante ponto a interação transformadora, permitindo um trabalho coletivo de diversos setores e em quaisquer áreas do conhecimento. Por isso, a Extensão Universitária pode ser uma excelente ferramenta na promoção da Inovação Social Aberta, aproveitando o capital humano de docentes, discentes e técnicos administrativos da Universidade para atuar diretamente em projetos de organizações da sociedade civil que visem o atendimento das necessidades dessas organizações.
