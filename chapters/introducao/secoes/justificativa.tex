\section{Justificativa}
\label{justificativa}

“Em países em desenvolvimento, torna-se pertinente analisar inovações […], com potencial de solucionar  problemas que afligem a sociedade, bem como […] entender a  interação das universidades com atores sociais diversos e seu papel.” (\citeauthor{klaumann2023} \citeyear{klaumann2023}).

Segundo \citeauthor{calefato2016} (\citeyear{calefato2016}), as \gls{ONG}s frequentemente vivenciam situações de escassez, tanto em seus recursos humanos, quanto financeiros. Visando auxiliar as \gls{ONG}s no atendimento de algumas de suas necessidades e finalidades, unindo a Extensão Universitária, e considerando o cenário desafiador das práticas extensionistas e das complexidades enfrentadas pelas ONGs, a Inovação Social Aberta pode ser o elo que irá fortalecer a Universidade e a Sociedade, sendo um catalisador e propulsor destas. 

Existe um benefício nítido por parte das organizações corporativas e governamentais da utilização da Inovação Aberta, porém, a quantidade de iniciativas voltada para as organizações do terceiro setor, que de fato utilizam os princípios da Inovação Social Aberta ainda é baixa. \cite{gama2023}

A universidade necessita evidenciar para a sociedade que seus benefícios vão além dos seus limites institucionais e da formação acadêmica. Para isso deve se colocar à disposição da comunidade, por meio de sua terceira missão, como: promoção de recursos para a comunidade, transmissão de conhecimento (ou troca de conhecimentos), prestação de serviços, soluções para os problemas da sociedade, e agir em nome da comunidade. (Benneworth (2012) apud \cite{klaumann2023})

Segundo \citeauthor{klaumann2023} (\citeyear{klaumann2023}), na literatura, ainda existem poucos artigos que tratam da inovação social e desse diálogo e interação entre a academia e a sociedade, sendo assim, é necessário estudar como ocorre a Inovação Social Aberta nas universidades, se existem práticas através da Extensão Universitária, compreender as nuances e desafios e propor posteriormente o início da implantação de processos que possam potencializar esse processo para ambos atores.

