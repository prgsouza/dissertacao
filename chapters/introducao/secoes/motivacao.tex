\section{Motivação}
\label{motivacao}

Segundo \citeauthor{pinheiro2020} (\citeyear{pinheiro2020}), a Inovação Social já possui diversas iniciativas de prática, porém, cada uma vem sendo praticada isoladamente, onde não se dialoga com atores externos a organização social que pratica a Inovação Social, tornando assim a prática mais complexa. Em decorrência disto, surge a necessidade de se realizar a prática através da atuação de outros atores externos as organizações sociais, que podem colaborar no aprimoramento e desenvolvimento das soluções desejadas pelas organizações, tendendo a obter um ganho quantitativo e qualitativo, ao contrário das práticas não-colaborativas. Isto pode ser aplicado nas organizações do terceiro setor, que, como dito por \citeauthor{gama2023} (\citeyear{gama2023}) possuem poucos recursos humanos e financeiros para executarem exitosamente suas atividades inovativas.

Um exemplo de sucesso de prática de Inovação Social Aberta no Brasil é o Porto Social, o qual é uma organização que atua como incubadora e aceleradora de iniciativas de organizações da sociedade civil, e parte destas organizações são \gls{ONG}s. Seu objetivo é potencializar o impacto social e criar soluções inovadoras. Apesar do Porto Social encubar projetos de diversas organizações da sociedade civil, parte das organizações que se classificam em seus editais, são ONGs. E esse número revela que existe demanda social para organização e auxílio destes projetos para atendimento das necessidades destas mesmas organizações, principalmente, de capital humano e intelectual, para o desenvolvimento e aprimoramento dos seus processos. A Universidade pode auxiliar no provimento desse capital, através da Extensão Universitária. \cite{portosocial2023}

Segundo a Política Nacional de Extensão Universitária do \gls{FORPROEX} (\citeyear{forproex2016}), a Extensão Universitária é um ponto indispensável na formação acadêmica dos estudantes, na qualificação dos docentes e também na troca de saberes da academia com a sociedade, numa via de mão dupla, onde a universidade também aprende com a sociedade, e os dois lados atuam para potencializar a resolução de importantes e complexos problemas do contexto social, de uma forma interprofissional e transdisciplinar.


%\vspace{1cm}

