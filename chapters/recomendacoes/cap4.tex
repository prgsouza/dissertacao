\section*{\#R4 – Promover Escuta Ativa e Aproximação com a Instituição}

O momento de escuta ativa é um elemento central no processo da Inovação Social Aberta. Esse momento de empatia permite que os estudantes compreendam melhor as dores, desafios e realidade do público com o qual irão trabalhar colaborativamente.

Nesse mesmo campo, a escuta ativa pode trazer uma valorização e compreensão maior por parte da instituição que trabalhará em colaboração com a universidade, potencializando a relação dialógica. Isso permite a criação de um espaço seguro de diálogo horizontal, imprescindível para a Inovação Social Aberta.

\textbf{Boas práticas:}
\begin{itemize}
    \item Local do encontro:
    \begin{itemize}
            \item Instituição parceira: fortalece a imersão e o processo de empatia;
            \item Espaços da universidade: possui uma logística mais facilitada por ser um ambiente familiar para os estudantes, mas pode enfraquecer o processo de vivência.
    \end{itemize}

    \item Metodologias do encontro:
    \begin{itemize}
            \item Hackathon: esse tipo de encontro irá enfatizar a construção de protótipos e mínimos produtos viáveis, em curtos espaços de tempo. É um momento de construção e criação. Pode facilitar na concepção rápida do protótipo, porém, pode propiciar a atuação num escopo menor, e também exige habilidades técnicas de seus participantes.
            \item Ideathon: nesse caso, existe a ênfase na concepção de ideias criativas e soluções inovadoras para problemas específicos, podendo resultar em protótipos ou não, pois seu foco não é a construção e criação, mas sim o pensamento e a conceituação, necessitando assim de outras etapas para a construção dessas soluções.
    \end{itemize}

    \item Gestão de tempo:
    \begin{itemize}
            \item Cada um dos momentos do encontro devem possuir seu tempo máximo de execução, e preferencialmente, serem cronometrados para maior controle;
            \item Importante evitar que relatos que não colaboram ativamente com o desenvolvimento do projeto desviem o foco, porém, isso deve ser feito de maneira empática e respeitosa;
    \end{itemize}
\end{itemize}

Em algumas situações, dependendo da dimensão da equipe, o processo de ideação e seleção de ideias pode ocorrer durante a escuta ativa. Nesses casos, metodologias formais tornam-se dispensáveis, sendo necessárias somente nas metodologias de levantamento e refinamento de requisitos e prototipação.

Os estudantes tendem a se sentir mais motivados e participativos quando conseguem se colocar no lugar das pessoas que passam pelas dificuldades. Alguns relatam experiências pessoais, fortalecendo laços entre si e aumentando a motivação.
