\chapter{Recomendações para a prática da Inovação Social Aberta na Extensão Universitária.}
\label{chap:recomendacoes}

Essas recomendações, como resultado desta dissertação, visam trazer práticas experimentadas, replicáveis e adaptáveis de Inovação Social Aberta na Extensão Universitária, mais especificamente voltadas para a concepção de artefatos digitais como meio de transformação social. Ele é destinado a docentes, discentes e gestores de organizações do terceiro setor que possuem interesse na implementação de ações colaborativas e dialógicas. Os dados analisados para constar neste guia são de uma revisão rápida de literatura acerca da produção acadêmica relevante sobre o tema da Inovação Social Aberta praticada por meio de universidades, do Projeto de Extensão CInbora Impactar, executado durante a disciplina de Desenvolvimento de Software, do curso de Sistemas de Informação do Centro de Informática da UFPE, em parceria com a Prefeitura da Cidade do Recife e do Projeto de Extensão CInovação Social, executado com estudantes do curso de Sistemas de Informação do Centro de Informática da UFPE em parceria com a ONG Gris Solidário da Várzea, Recife–PE.

As recomendações foram organizadas agrupadamente, e na observação dos projetos foram compiladas as seguintes:

\begin{table}[H]
\centering
\caption{Recomendações para a Extensão Universitária}
\begin{tabular}{|c|p{11cm}|}
\hline
\textbf{Recomendação} & \textbf{Descrição} \\ \hline
1 & Utilizar a Inovação Social Aberta na Extensão \\ \hline
2 & Planejar o Projeto de Extensão  \\ \hline
3 & Promover escuta ativa, respeito e empatia com a instituição \\ \hline
4 & Valorizar processos de aprendizagem e desenvolvimento com \textit{soft-skills} \\ \hline
5 & Utilizar metodologias e ferramentas de comunicação \\ \hline
6 & Utilizar ferramentas de gerenciamento de tarefas \\ \hline
7 & Empregar técnicas de ideação e geração de soluções \\ \hline
8 & Desenvolver colaborativamente artefatos digitais adequadamente \\ \hline
9 & Avaliar as soluções propostas e o aproveitamento dos estudantes \\ \hline
\end{tabular}
\end{table}



\section*{\#R1 – Utilizar a Inovação Social Aberta na Extensão Universitária}

A Extensão Universitária é um importante e poderoso instrumento da universidade que possui um enorme potencial de transformação social. Entre as suas diretrizes está a Interação Dialógica, que apregoa a participação ativa e a construção conjunta de conhecimento, valorizando o diálogo entre diferentes setores da sociedade e a universidade.

Para além da função social da Extensão Universitária, a mesma também atua como uma propagandista do trabalho executado pela Universidade, onde muitas vezes pode despertar na comunidade, que não conhece profundamente a Universidade e o seu trabalho, a vontade de ingressar nas fileiras acadêmicas.

A Inovação Social Aberta pode auxiliar em todos esses processos, por ter como seu cerne a Inovação Aberta, a qual é a troca de conhecimento, através da recepção de ideias externas, exportação de ideias e cocriação para potencialização do processo inovativo, em vez da utilização exclusiva dos próprios recursos. A Inovação Social busca criar ou aprimorar ideias, práticas, serviços, dentre outros, para atender às necessidades das comunidades e proporcionar melhores condições de vida para as mesmas.

A utilização da Inovação Social Aberta para potencializar os processos extensionistas, acaba por auxiliar a universidade a fortalecer sua capacidade, em conjunto com a comunidade, de identificar, co-criar e implementar soluções inovadoras para seus problemas reais. Esse processo não auxilia apenas o cotidiano da comunidade onde a instituição está inserida, mas também reforça o papel da universidade pública para com a comunidade na qual está inserida. Esse fortalecimento é importante para demonstrar a verdadeira essência e importância das universidades públicas, principalmente em cenários de cortes orçamentários, que são recorrentes no cenário brasileiro, como apontado por \citeauthor{spannenberg2023} (\citeyear{spannenberg2023}).
\section*{\#R2 – Planejamento do Projeto}

Um bom planejamento é extremamente importante para o sucesso de qualquer projeto, especialmente em projetos de Inovação Social Aberta por meio da extensão universitária, onde existe o envolvimento de diversos atores diferentes em sua composição. Ele vai garantir a coesão de todos os atores e equipes, delimitação de objetivos, organização de cada etapa a ser executada e uma melhor assertividade no atendimento destes objetivos.

\textbf{Definição de escopo e objetivos}:
Um dos principais pontos a ser observado ao realizar o planejamento do projeto é a definição e delimitação do escopo que será atendido e dos objetivos que serão trabalhados. Como percebido no projeto CInovação Social, as instituições, ao observarem a oportunidade de construírem algo em conjunto com a universidade, pode acabar querendo expandir o escopo do projeto, o que pode acabar sendo inviável, ao depender do projeto e de sua duração. É importante que o coordenador junto com a equipe delimite antes do início da execução do mesmo o seu escopo e os seus objetivos, além dos resultados que buscam ser alcançados, sempre focando na formação dos estudantes, nos benefícios vivenciados pela instituição e no fortalecimento do processo de relação dialógica entre a universidade e a comunidade.

Com o advento da curricularização da extensão nas universidades, cada vez mais têm se oferecido disciplinas já integradas com carga horária de extensão universitária, porém, cada projeto de extensão terá sua própria dinâmica, que poderá se adequar mais a um tipo específico de configuração de extensão.

Os projetos de extensão podem ocorrer integrados a uma disciplina ou de maneira autônoma.

\textbf{Projetos integrados a disciplinas (ex.: CInbora Impactar):}
\begin{itemize}
    \item Conteúdo teórico aplicado simultaneamente.
    \item Projetos reais em vez de simulações.
    \item Maior engajamento dos estudantes na disciplina e/ou projeto.
\end{itemize}

\textbf{Pontos negativos:}
\begin{itemize}
    \item Pode competir com outras disciplinas do mesmo semestre.
    \item Um excesso de equipes pode dificultar o gerenciamento.
    \item Participação pode se tornar impessoal devido à quantidade de estudantes.
\end{itemize}

\textbf{Projetos autônomos (ex.: CInovação Social):}
\begin{itemize}
    \item Equipes menores podem gerar participação mais pessoal e empática.
    \item Tempo de duração variável, podendo incluir períodos de férias.
\end{itemize}

\textbf{Pontos negativos:}
\begin{itemize}
    \item Possível desmotivação por ausência de impacto acadêmico perceptível a curto prazo.
    \item Dificuldade no financiamento das atividades.
\end{itemize}

\textbf{Duração do projeto e carga horária:}
Cada projeto de extensão, desde sua concepção e observando os seus objetivos e resultados esperados, deve possuir sua duração delimitada pelo coordenador e equipe com base na experiência vivenciada e pela mensuração das atividades e do tempo. Alguns dos cenários possíveis são os seguintes:

\begin{itemize}
    \item \textbf{Curto prazo}: Este cenário pode ser utilizado para projetos com menos de um semestre de duração, e é mais adequado para projetos que envolvem metodologias ágeis e prototipagem rápida;
    \item \textbf{Médio prazo}: Pode ser utilizado em disciplinas, com duração de um semestre (podendo se estender a dois semestres, se for um projeto mais longo), e é um cenário mais adequado para a curricularização da extensão;
    \item \textbf{Longo prazo}: É mais indicado para projetos estruturantes, que exija uma maior necessidade de interação entre os estudantes e a instituição, e a necessidade de diversas iterações para atingir o resultado desejado.
\end{itemize}

\textbf{Delimitação das metodologias:}
Para apoiar a definição do escopo, objetivos e resultados esperados, é importante que a metodologia que será utilizada no projeto seja bem delimitada e documentada, sempre com fases claras, e quais serão os entregáveis esperados pelo projeto. Ao longo das recomendações existem metodologias e entregáveis que podem ser utilizados em projetos de extensão que visam o desenvolvimento de artefatos digitais. 




\section*{\#R3 – Promover Escuta Ativa, Respeito e Empatia com a Instituição}

O momento de escuta ativa é um elemento central no processo da Inovação Social Aberta. Esse momento de empatia permite que os estudantes compreendam melhor as dores, desafios e realidade do público com o qual irão trabalhar colaborativamente.

Nesse mesmo campo, a escuta ativa pode trazer uma valorização e compreensão maior por parte da instituição que trabalhará em colaboração com a universidade, potencializando a relação dialógica. Isso permite a criação de um espaço seguro de diálogo horizontal, imprescindível para a Inovação Social Aberta.

\textbf{Boas práticas:}
\begin{itemize}
    \item Local do encontro:
    \begin{itemize}
            \item Instituição parceira: fortalece a imersão e o processo de empatia;
            \item Espaços da universidade: possui uma logística mais facilitada por ser um ambiente familiar para os estudantes, mas pode enfraquecer o processo de vivência.
    \end{itemize}

    \item Metodologias do encontro:
    \begin{itemize}
            \item Hackathon: esse tipo de encontro irá enfatizar a construção de protótipos e mínimos produtos viáveis, em curtos espaços de tempo. É um momento de construção e criação. Pode facilitar na concepção rápida do protótipo, porém, pode propiciar a atuação num escopo menor, e também exige habilidades técnicas de seus participantes.
            \item Ideathon: nesse caso, existe a ênfase na concepção de ideias criativas e soluções inovadoras para problemas específicos, podendo resultar em protótipos ou não, pois seu foco não é a construção e criação, mas sim o pensamento e a conceituação, necessitando assim de outras etapas para a construção dessas soluções.
    \end{itemize}

    \item Gestão de tempo:
    \begin{itemize}
            \item Cada um dos momentos do encontro devem possuir seu tempo máximo de execução, e preferencialmente, serem cronometrados para maior controle;
            \item Importante evitar que relatos que não colaboram ativamente com o desenvolvimento do projeto desviem o foco, porém, isso deve ser feito de maneira empática e respeitosa;
    \end{itemize}
\end{itemize}

Em algumas situações, dependendo da dimensão da equipe, o processo de ideação e seleção de ideias pode ocorrer durante a escuta ativa. Nesses casos, metodologias formais tornam-se dispensáveis, sendo necessárias somente nas metodologias de levantamento e refinamento de requisitos e prototipação.

Os estudantes tendem a se sentir mais motivados e participativos quando conseguem se colocar no lugar das pessoas que passam pelas dificuldades. Alguns relatam experiências pessoais, fortalecendo laços entre si e aumentando a motivação.

A extensão universitária exige um esforço de empatia por parte dos estudantes, para compreenderem os reais desafios que as instituições do terceiro setor vivenciam. Elas não são empresas que possuem horários rígidos, cronogramas inflexíveis, mas sim um organismo vivo que possui inúmeros desafios diários, além de vivenciarem um quantitativo extremamente reduzido de recursos humanos e recursos financeiros, como apontado por \citeauthor{gama2023} (\citeyear{gama2023}). O respeito e empatia deste momento é extremamente importante para o fortalecer o processo da relação dialógica.

No projeto CInovação social, ocorreu uma situação na qual a representante da instituição necessitou remarcar um dos encontros para participar de um evento para captação de recursos para a própria instituição. 

Outra situação foi de algumas interrupções que ocorreram no \textit{Status Report}, por parte da representante da instituição, que necessitou de dar uma rápida atenção para seu filho pequeno, ao mesmo tempo que participava da reunião e se desdobrava para resolver outras questões. 

Essas situações exigem um nível de empatia e compreensão dos estudantes, e da equipe executora do projeto e também são experiências valiosas, por colocarem os estudantes confrontando a realidade vivenciada pelas instituições do terceiro setor, onde muitos dos seus colaboradores acabam executando diversas tarefas nos mais diversos escopos de atuação, por conta da necessidade e da falta de voluntários para os auxiliar.

Quem está liderando o projeto deve estimular a empatia e o respeito por parte dos estudantes e de todo o grupo, e sempre ressaltar que essas situações não devem ser entendidas como uma falta de compromisso por parte da instituição, mas sim uma parte pulsante e viva da instituição.

\section*{\#R4 – Agir com respeito e empatia com o espaço e tempo da instituição parceira}

A extensão universitária exige um esforço de empatia por parte dos estudantes, para compreenderem os reais desafios que as instituições do terceiro setor vivenciam. Elas não são empresas que possuem horários rígidos, cronogramas inflexíveis, mas sim um organismo vivo que possui inúmeros desafios diários, além de vivenciarem um quantitativo extremamente reduzido de recursos humanos e recursos financeiros, como apontado por \citeauthor{gama2023} (\citeyear{gama2023}). O respeito e empatia deste momento é extremamente importante para o fortalecer o processo da relação dialógica.

No projeto CInovação social, ocorreu uma situação na qual a representante da instituição necessitou remarcar um dos encontros para participar de um evento para captação de recursos para a própria instituição. 

Outra situação foi de algumas interrupções que ocorreram no \textit{Status Report}, por parte da representante da instituição, que necessitou de dar uma rápida atenção para seu filho pequeno, ao mesmo tempo que participava da reunião e se desdobrava para resolver outras questões. 

Essas situações exigem um nível de empatia e compreensão dos estudantes, e da equipe executora do projeto e também são experiências valiosas, por colocarem os estudantes confrontando a realidade vivenciada pelas instituições do terceiro setor, onde muitos dos seus colaboradores acabam executando diversas tarefas nos mais diversos escopos de atuação, por conta da necessidade e da falta de voluntários para os auxiliar.

Quem está liderando o projeto deve estimular a empatia e o respeito por parte dos estudantes e de todo o grupo, e sempre ressaltar que essas situações não devem ser entendidas como uma falta de compromisso por parte da instituição, mas sim uma parte pulsante e viva da instituição.

\section*{\#R5 – Utilizar Metodologias e Ferramentas de Comunicação}

Um ponto extremamente importante para o bom desenvolvimento do projeto é a comunicação, tanto interna entre a equipe executora do projeto, quanto com a instituição parceira. 

Diversas ferramentas podem ser utilizadas para comunicação em projetos de extensão. As que se demonstraram efetivas são:
\begin{itemize}
    \item WhatsApp: comunicação informal e atualizações rápidas. É importante que o WhatsApp, por ser um meio de comunicação mais difundido na sociedade, seja um espaço de utilização tanto pelos estudantes quanto pela organização. Uma boa forma de delimitar isto é por meio de um grupo de WhatsApp com todos os envolvidos no projeto, porém, delimitando que aquele espaço é exclusivamente para tratar de assuntos relacionados ao projeto, para evitar discussões paralelas sem relação com a extensão.
    \item Discord: utilizado para codificação conjunta, resolução de problemas, agendamento de reuniões, enquetes e videochamadas. É um aplicativo mais específico e de mais conhecimento dos estudantes. Pode ser mais interessante para uso interno da equipe, sem envolver a instituição parceira.
\end{itemize}
\section*{\#R6 – Definir a Duração do Projeto com Metodologias Ágeis}

É imprescindível que o projeto possua as suas etapas bem definidas. A falta dessa definição pode gerar um tom de desorganização por parte do coordenador do projeto, o que pode causar ansiedade em alguns estudantes ou até mesmo desestimular os mesmos. 

Uma possível metodologia para projetos de extensão e inovação social é a metodologia Speedplay de \citeauthor{ferrario2014} (\citeyear{ferrario2014}), o qual é um método de desenvolvimento de inovação social por meio de artefatos digitais, em ambientes que exigem uma maior celeridade, além de grupos com difícil acesso, representando uma metodologia de execução rápida, que permite uma autonomia na forma organizacional para quem irá realizar a coordenação do projeto. Como possui o foco em artefatos digitais e prototipagem rápida e exige um ritmo acelerado, o Speedplay funciona bem com programação via Inteligência Artificial (Vibecoding), que será falado nos próximos capítulos, onde o Speedplay atua como bússola metodológica, enquanto o Vibecoding traduz essa bússola em código computacional. 

Um dos momentos importantes do Speedplay é o marco chamado por Ferrario de Ponto focal, onde será um momento de aceleração da colaboração e um importante momento de engajamento. As formas de execução desse ponto focal serão mais detalhadas nos próximos capítulos.

As fases do Speedplay são as seguintes:

\begin{enumerate}
    \item Preparar: \textit{ideathon} ou \textit{hackathon} como pontos focais, levantamento de requisitos, início de protótipos.
    \item Co-Criar: exploração das ideias e desenho de soluções.
    \item Construir: construção de MVP, testes e validação.
    \item Sustentar: consolidação do aprendizado e continuidade das soluções.
\end{enumerate}

Em projetos curtos, o Scrum pode ser adaptado para “mini-sprints”, exigindo encontros frequentes de \textit{Sprint Reviews} e testes de usuários.

\section*{\#R7 – Valorizar Processos de Aprendizagem e Engajamento e Desenvolvimento de Soft-Skills}

O processo de ensino-aprendizagem é uma das partes mais importantes da extensão. Muitas das vezes se existe um grande foco na solução que será entregue, mas esta solução será somente um reflexo de todo o processo de ensino-aprendizagem vivenciado tanto pelos estudantes quanto pela instituição parceira no processo de cocriação e de inovação aberta.

O processo de inovação social aberta ocorre através da troca de aprendizado e da relação dialógica:
\begin{itemize}
    \item A instituição aprende com os estudantes.
    \item Os estudantes integram teoria e prática em experiências reais e significativas.
\end{itemize}

Isso potencializa o desenvolvimento de \textit{soft-skills}, como resolução de problemas, gerenciamento de times e pensamento crítico. É importante resgatar o protagonismo estudantil, tornando os projetos autogerenciáveis. O coordenador, entretanto, deve acompanhar constantemente, garantindo que o projeto funcione e os estudantes não se sintam somente força de trabalho.

Ao usar metodologias ágeis, funções (ex.: \textit{Product Owner}, \textit{Scrum Master}) devem ser definidas desde o início, com apoio do coordenador no desenvolvimento de habilidades gerenciais. O coordenador também pode atuar diretamente na execução, incluindo a confecção do artefato digital, gerando maior engajamento e segurança nos estudantes.

O registro de todas as atividades e decisões é fundamental para auxiliar na execução do projeto e para lições aprendidas em projetos futuros.

\section*{\#R8 – Desenvolver colaborativamente Artefatos Digitais}

Segundo o preconizado pela metodologia \textit{Speedplay} de \citeauthor{ferrario2014} (\citeyear{ferrario2014}), os artefatos digitais podem ser utilizados como meio de transformação social, o qual é o foco deste trabalho. 

Existem diversas formas de construção de artefatos digitais em projetos de extensão, e as formas vivenciadas e acertadas, que ocorreram nos projetos do Centro de Informática avaliados por essas recomendações, são as seguintes:

\begin{itemize}
    \item Programação manual: os estudantes desenvolvem todos os códigos e interfaces dos artefatos digitais. Recomendado para equipes grandes e períodos de execução mais longos.
    \item Programação via Inteligência Artificial (\textit{vibecoding}): os estudantes desenvolvem códigos e interfaces por meio da inteligência artificial, realizando manualmente correções e adaptações necessárias. Recomendado para equipes menores e períodos de execução mais curtos, permitindo maior foco nas ideias, processos e entendimento da dor da instituição.
\end{itemize}

Como a forma da realização da programação manual pode se dar de diversas formas, utilizando as mais diversas metodologias e \textit{frameworks} que já estão consolidados e são bastante conhecidos pela literatura científica, este trabalho vai trazer uma nova perspectiva, que está em amplo crescimento, e trazer algumas recomendações básicas de como utilizar exitosamente, o qual é o \textit{vibecoding}.

A forma de programação por meio de Inteligência Artificial tem se mostrado em evidência, em decorrência do crescimento súbito do tema no mercado e na mídia, o que também pode ser um motivador maior para os estudantes, por terem a oportunidade de trabalharem com novas tecnologias que estão cada vez mais em evidência no cotidiano do mercado. 

Ferramentas úteis no Vibecoding:
\begin{itemize}
    \item \textbf{ChatGPT}: visando uma melhor assertividade da utilização dos créditos gratuitos disponibilizados pelo V0/Vercel, os estudantes utilizaram o ChatGPT para melhor criação do \textit{prompt} conforme os requisitos levantados em conjunto com a ONG e com a interface que foi prototipada, seja no Wireframe ou no FIGMA, onde o ChatGPT permite a adição de fotos, facilitando na construção do \textit{prompt};

    Exemplo de \textit{prompt} utilizado no projeto CInovação Social: Crie um prompt para o v0 com base nesse projeto (descrição básica do projeto) e nesse protótipo (enviado imagem do protótipo do FIGMA).
    
    \item \textbf{V0 - Vercel}: as interfaces (front-end) do projeto CInovação Social e seu banco de dados foi todo gerado via IA através do V0, onde os estudantes, através do \textit{prompt} gerado pelo ChatGPT, gerou o código em React/Next.js com Tailwind, entregando componentes reutilizáveis, e também gerando o \textit{back-end} e o banco de dados completo no Supabase, além de realizar automaticamente a integração destes 3 componentes da plataforma. Em decorrência do V0 ser pago e possuir poucos créditos gratuitos, foi utilizado para gerar o núcleo duro do artefato, onde os estudantes poderão trabalhar em cima do projeto, refinando o mesmo. O tempo economizado durante a construção dos principais componentes necessários para a solução pode ser utilizado para a documentação e melhoria incremental do projeto.
    
    \item \textbf{Gemini}: diversos erros foram ocorrendo durante o processo de criação da solução através do V0/Vercel, e para a correção dos erros, seja de interface ou do funcionamento na lógica, cada correção era cobrada a parte pela plataforma, e muitas das vezes as soluções eram incompletas, gerando mais cobranças para novas correções. Em vez de realizar as correções pelo V0/Vercel, os erros e os códigos foram enviados para o Gemini, onde o mesmo realizava a proposta de correção, que era analisada pelos estudantes e era integrada de fato no projeto. Alguns erros se repetiam em diferentes módulos do sistema, e ao corrigir algum caso específico junto ao Gemini, os estudantes conseguiram otimizar o tempo aplicando as mesmas correções em outros módulos, gerando também um aprendizado coletivo sobre padrões de erro e correções no código.
    
    \item \textbf{Copilot}: essa ferramenta foi utilizada durante o processo de codificação por parte dos estudantes, onde a correção de erros do V0/Vercel e do Gemini não foi efetiva, e precisava de um olhar mais detalhado pelos estudantes, através da utilização de ferramentas de codificação comuns como o Visual Studio Code, em auxilio do Copilot para correções mais pontuais via IA.
\end{itemize}

É importante garantir que boas práticas estão sendo adotadas durante o desenvolvimento colaborativo, que muitas vezes pode ser desafiador por envolver estudantes de períodos diferentes, cursos diferentes, e realidades diferentes, além do envolvimento da instituição parceira, que pode ter limitações que dificulte sua participação neste desenvolvimento.

Algumas práticas foram adotadas ao longo das execuções dos projetos analisados:
\begin{itemize}
    \item Dias fixos de trabalho: o \textit{Product Owner} do projeto pode definir dias e horários fixos para todos desenvolverem colaborativamente, evitando que o único momento de encontro entre toda a equipe seja no \textit{Sprint Review}. Isso pode auxiliar e estimular a responsabilidade dos participantes e o engajamento dos mesmos. Em casos de projetos que estão sendo executados em períodos de férias, pode se dar uma preferência aos horários já reservados para as aulas, auxiliando assim a organização dos estudantes. Em projetos executados em períodos de aula, podem ser ao fim de semana, respeitando o descanso dos estudantes. Esses encontros podem ser tanto presenciais quanto \textit{online}. O uso da ferramenta Discord no projeto CInovação Social permitiu a otimização destes momentos através dos canais de voz, onde os estudantes poderiam conversar entre si por voz, compartilhar suas telas para mostrarem o que estavam fazendo, além de poderem enviar código via chat, acelerando o processo.
    
    \item Delimitação de tarefas: deve se ter uma forte delimitação dos papéis que serão executados pelos estudantes e pelo coordenador do projeto, além dos papéis que serão exercidos pelos colaboradores da instituição parceira. Caso adotada metodologias como o Scrum, devem ser delimitados os papéis básicos como \textit{Product Owner}, \textit{Scrum Master}. Também deve haver o direcionamento e a separação de tarefas da concepção e do desenvolvimento do artefato digital, evitando sobreposição de trabalho e retrabalho entre os estudantes, além de deixar os mesmos mais confortáveis de trabalharem nas áreas que se sentem mais capacitados.

    \item Feedback contínuo: é importante que o coordenador estimule o \textit{feedback} contínuo em relação às tarefas que os estudantes estão executando ao longo do desenrolar do projeto, tanto do coordenador para com os estudantes, dos estudantes para consigo mesmos e dos estudantes para o coordenador, auxiliando na melhoria do projeto não somente em cada nova iteração, mas ao longo da própria execução do mesmo.

\end{itemize}
\section*{\#R9 – Utilizar Ferramentas de Gerenciamento de Tarefas}

O gerenciamento de tarefas é fundamental para o sucesso de projetos. Um bom planejamento e gerenciamento, além de reduzir a quantidade de erros que podem ocorrer no processo e na solução final, também evita retrabalho e permite uma otimização do tempo de todos os participantes.

Essas são algumas ferramentas básicas que podem auxiliar no gerenciamento das tarefas no projeto:

\begin{itemize}
    \item Jira: especializado em gestão de projetos de software, com rastreamento de bugs e relatórios.
    \item Notion: versátil, permite bases de dados, listas de tarefas e organização visual.
    \item Trello: simples e eficaz com quadros Kanban para backlog.
\end{itemize}

Apesar do uso dessas ferramentas, muitas vezes a comunicação informal acaba predominando como principal forma de gerenciamento. Essa não é a forma ideal, visto que não permite a documentação do projeto, o que pode dificultar para a posterior avaliação dos pontos críticos, o levantamento de lições aprendidas, e a consequente melhoria nas próximas execuções do projeto.










