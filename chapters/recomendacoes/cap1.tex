\section*{\#R1 – Utilizar a Inovação Social Aberta na Extensão Universitária}

A Extensão Universitária é um importante e poderoso instrumento da universidade que possui um enorme potencial de transformação social. Entre as suas diretrizes está a Interação Dialógica, que apregoa a participação ativa e a construção conjunta de conhecimento, valorizando o diálogo entre diferentes setores da sociedade e a universidade.

Para além da função social da Extensão Universitária, a mesma também atua como uma propagandista do trabalho executado pela Universidade, onde muitas vezes pode despertar na comunidade, que não conhece profundamente a Universidade e o seu trabalho, a vontade de ingressar nas fileiras acadêmicas.

A Inovação Social Aberta pode auxiliar em todos esses processos, por ter como seu cerne a Inovação Aberta, a qual é a troca de conhecimento, através da recepção de ideias externas, exportação de ideias e cocriação para potencialização do processo inovativo, em vez da utilização exclusiva dos próprios recursos. A Inovação Social busca criar ou aprimorar ideias, práticas, serviços, dentre outros, para atender às necessidades das comunidades e proporcionar melhores condições de vida para as mesmas.

A utilização da Inovação Social Aberta para potencializar os processos extensionistas, acaba por auxiliar a universidade a fortalecer sua capacidade, em conjunto com a comunidade, de identificar, co-criar e implementar soluções inovadoras para seus problemas reais. Esse processo não auxilia apenas o cotidiano da comunidade onde a instituição está inserida, mas também reforça o papel da universidade pública para com a comunidade na qual está inserida. Esse fortalecimento é importante para demonstrar a verdadeira essência e importância das universidades públicas, principalmente em cenários de cortes orçamentários, que são recorrentes no cenário brasileiro, como apontado por \citeauthor{spannenberg2023} (\citeyear{spannenberg2023}).