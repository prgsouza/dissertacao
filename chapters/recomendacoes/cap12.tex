\section*{\#R12 – Adotar Boas Práticas de Desenvolvimento Colaborativo}

É importante garantir que boas práticas estão sendo adotadas durante o desenvolvimento colaborativo, que muitas vezes pode ser desafiador por envolver estudantes de períodos diferentes, cursos diferentes, e realidades diferentes, além do envolvimento da instituição parceira, que pode ter limitações que dificulte sua participação neste desenvolvimento.

Algumas práticas foram adotadas ao longo das execuções dos projetos analisados:
\begin{itemize}
    \item Dias fixos de trabalho: o \textit{Product Owner} do projeto pode definir dias e horários fixos para todos desenvolverem colaborativamente, evitando que o único momento de encontro entre toda a equipe seja no \textit{Sprint Review}. Isso pode auxiliar e estimular a responsabilidade dos participantes e o engajamento dos mesmos. Em casos de projetos que estão sendo executados em períodos de férias, pode se dar uma preferência aos horários já reservados para as aulas, auxiliando assim a organização dos estudantes. Em projetos executados em períodos de aula, podem ser ao fim de semana, respeitando o descanso dos estudantes. Esses encontros podem ser tanto presenciais quanto \textit{online}. O uso da ferramenta Discord no projeto CInovação Social permitiu a otimização destes momentos através dos canais de voz, onde os estudantes poderiam conversar entre si por voz, compartilhar suas telas para mostrarem o que estavam fazendo, além de poderem enviar código via chat, acelerando o processo.
    
    \item Delimitação de tarefas: deve se ter uma forte delimitação dos papéis que serão executados pelos estudantes e pelo coordenador do projeto, além dos papéis que serão exercidos pelos colaboradores da instituição parceira. Caso adotada metodologias como o Scrum, devem ser delimitados os papéis básicos como \textit{Product Owner}, \textit{Scrum Master}. Também deve haver o direcionamento e a separação de tarefas da concepção e do desenvolvimento do artefato digital, evitando sobreposição de trabalho e retrabalho entre os estudantes, além de deixar os mesmos mais confortáveis de trabalharem nas áreas que se sentem mais capacitados.

    \item Feedback contínuo: é importante que o coordenador estimule o \textit{feedback} contínuo em relação às tarefas que os estudantes estão executando ao longo do desenrolar do projeto, tanto do coordenador para com os estudantes, dos estudantes para consigo mesmos e dos estudantes para o coordenador, auxiliando na melhoria do projeto não somente em cada nova iteração, mas ao longo da própria execução do mesmo.

\end{itemize}
