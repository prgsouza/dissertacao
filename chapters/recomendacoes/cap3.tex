\section*{\#R3 – Promover Escuta Ativa, Respeito e Empatia com a Instituição}

O momento de escuta ativa é um elemento central no processo da Inovação Social Aberta. Esse momento de empatia permite que os estudantes compreendam melhor as dores, desafios e realidade do público com o qual irão trabalhar colaborativamente.

Nesse mesmo campo, a escuta ativa pode trazer uma valorização e compreensão maior por parte da instituição que trabalhará em colaboração com a universidade, potencializando a relação dialógica. Isso permite a criação de um espaço seguro de diálogo horizontal, imprescindível para a Inovação Social Aberta.

\textbf{Boas práticas:}
\begin{itemize}
    \item Local do encontro:
    \begin{itemize}
            \item Instituição parceira: fortalece a imersão e o processo de empatia;
            \item Espaços da universidade: possui uma logística mais facilitada por ser um ambiente familiar para os estudantes, mas pode enfraquecer o processo de vivência.
    \end{itemize}

    \item Metodologias do encontro:
    \begin{itemize}
            \item Hackathon: esse tipo de encontro irá enfatizar a construção de protótipos e mínimos produtos viáveis, em curtos espaços de tempo. É um momento de construção e criação. Pode facilitar na concepção rápida do protótipo, porém, pode propiciar a atuação num escopo menor, e também exige habilidades técnicas de seus participantes.
            \item Ideathon: nesse caso, existe a ênfase na concepção de ideias criativas e soluções inovadoras para problemas específicos, podendo resultar em protótipos ou não, pois seu foco não é a construção e criação, mas sim o pensamento e a conceituação, necessitando assim de outras etapas para a construção dessas soluções.
    \end{itemize}

    \item Gestão de tempo:
    \begin{itemize}
            \item Cada um dos momentos do encontro devem possuir seu tempo máximo de execução, e preferencialmente, serem cronometrados para maior controle;
            \item Importante evitar que relatos que não colaboram ativamente com o desenvolvimento do projeto desviem o foco, porém, isso deve ser feito de maneira empática e respeitosa;
    \end{itemize}
\end{itemize}

Em algumas situações, dependendo da dimensão da equipe, o processo de ideação e seleção de ideias pode ocorrer durante a escuta ativa. Nesses casos, metodologias formais tornam-se dispensáveis, sendo necessárias somente nas metodologias de levantamento e refinamento de requisitos e prototipação.

Os estudantes tendem a se sentir mais motivados e participativos quando conseguem se colocar no lugar das pessoas que passam pelas dificuldades. Alguns relatam experiências pessoais, fortalecendo laços entre si e aumentando a motivação.

A extensão universitária exige um esforço de empatia por parte dos estudantes, para compreenderem os reais desafios que as instituições do terceiro setor vivenciam. Elas não são empresas que possuem horários rígidos, cronogramas inflexíveis, mas sim um organismo vivo que possui inúmeros desafios diários, além de vivenciarem um quantitativo extremamente reduzido de recursos humanos e recursos financeiros, como apontado por \citeauthor{gama2023} (\citeyear{gama2023}). O respeito e empatia deste momento é extremamente importante para o fortalecer o processo da relação dialógica.

No projeto CInovação social, ocorreu uma situação na qual a representante da instituição necessitou remarcar um dos encontros para participar de um evento para captação de recursos para a própria instituição. 

Outra situação foi de algumas interrupções que ocorreram no \textit{Status Report}, por parte da representante da instituição, que necessitou de dar uma rápida atenção para seu filho pequeno, ao mesmo tempo que participava da reunião e se desdobrava para resolver outras questões. 

Essas situações exigem um nível de empatia e compreensão dos estudantes, e da equipe executora do projeto e também são experiências valiosas, por colocarem os estudantes confrontando a realidade vivenciada pelas instituições do terceiro setor, onde muitos dos seus colaboradores acabam executando diversas tarefas nos mais diversos escopos de atuação, por conta da necessidade e da falta de voluntários para os auxiliar.

Quem está liderando o projeto deve estimular a empatia e o respeito por parte dos estudantes e de todo o grupo, e sempre ressaltar que essas situações não devem ser entendidas como uma falta de compromisso por parte da instituição, mas sim uma parte pulsante e viva da instituição.
