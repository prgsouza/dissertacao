\section*{\#R2 – Planejamento do Projeto}

Um bom planejamento é extremamente importante para o sucesso de qualquer projeto, especialmente em projetos de Inovação Social Aberta por meio da extensão universitária, onde existe o envolvimento de diversos atores diferentes em sua composição. Ele vai garantir a coesão de todos os atores e equipes, delimitação de objetivos, organização de cada etapa a ser executada e uma melhor assertividade no atendimento destes objetivos.

\textbf{Definição de escopo e objetivos}:
Um dos principais pontos a ser observado ao realizar o planejamento do projeto é a definição e delimitação do escopo que será atendido e dos objetivos que serão trabalhados. Como percebido no projeto CInovação Social, as instituições, ao observarem a oportunidade de construírem algo em conjunto com a universidade, pode acabar querendo expandir o escopo do projeto, o que pode acabar sendo inviável, ao depender do projeto e de sua duração. É importante que o coordenador junto com a equipe delimite antes do início da execução do mesmo o seu escopo e os seus objetivos, além dos resultados que buscam ser alcançados, sempre focando na formação dos estudantes, nos benefícios vivenciados pela instituição e no fortalecimento do processo de relação dialógica entre a universidade e a comunidade.

Com o advento da curricularização da extensão nas universidades, cada vez mais têm se oferecido disciplinas já integradas com carga horária de extensão universitária, porém, cada projeto de extensão terá sua própria dinâmica, que poderá se adequar mais a um tipo específico de configuração de extensão.

Os projetos de extensão podem ocorrer integrados a uma disciplina ou de maneira autônoma.

\textbf{Projetos integrados a disciplinas (ex.: CInbora Impactar):}
\begin{itemize}
    \item Conteúdo teórico aplicado simultaneamente.
    \item Projetos reais em vez de simulações.
    \item Maior engajamento dos estudantes na disciplina e/ou projeto.
\end{itemize}

\textbf{Pontos negativos:}
\begin{itemize}
    \item Pode competir com outras disciplinas do mesmo semestre.
    \item Um excesso de equipes pode dificultar o gerenciamento.
    \item Participação pode se tornar impessoal devido à quantidade de estudantes.
\end{itemize}

\textbf{Projetos autônomos (ex.: CInovação Social):}
\begin{itemize}
    \item Equipes menores podem gerar participação mais pessoal e empática.
    \item Tempo de duração variável, podendo incluir períodos de férias.
\end{itemize}

\textbf{Pontos negativos:}
\begin{itemize}
    \item Possível desmotivação por ausência de impacto acadêmico perceptível a curto prazo.
    \item Dificuldade no financiamento das atividades.
\end{itemize}

\textbf{Duração do projeto e carga horária:}
Cada projeto de extensão, desde sua concepção e observando os seus objetivos e resultados esperados, deve possuir sua duração delimitada pelo coordenador e equipe com base na experiência vivenciada e pela mensuração das atividades e do tempo. Alguns dos cenários possíveis são os seguintes:

\begin{itemize}
    \item \textbf{Curto prazo}: Este cenário pode ser utilizado para projetos com menos de um semestre de duração, e é mais adequado para projetos que envolvem metodologias ágeis e prototipagem rápida;
    \item \textbf{Médio prazo}: Pode ser utilizado em disciplinas, com duração de um semestre (podendo se estender a dois semestres, se for um projeto mais longo), e é um cenário mais adequado para a curricularização da extensão;
    \item \textbf{Longo prazo}: É mais indicado para projetos estruturantes, que exija uma maior necessidade de interação entre os estudantes e a instituição, e a necessidade de diversas iterações para atingir o resultado desejado.
\end{itemize}

\textbf{Delimitação das metodologias:}
Para apoiar a definição do escopo, objetivos e resultados esperados, é importante que a metodologia que será utilizada no projeto seja bem delimitada e documentada, sempre com fases claras, e quais serão os entregáveis esperados pelo projeto. Ao longo das recomendações existem metodologias e entregáveis que podem ser utilizados em projetos de extensão que visam o desenvolvimento de artefatos digitais. 



