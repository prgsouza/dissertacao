\section*{\#R5 – Utilizar Metodologias e Ferramentas de Comunicação}

Um ponto extremamente importante para o bom desenvolvimento do projeto é a comunicação, tanto interna entre a equipe executora do projeto, quanto com a instituição parceira. 

Diversas ferramentas podem ser utilizadas para comunicação em projetos de extensão. As que se demonstraram efetivas são:
\begin{itemize}
    \item WhatsApp: comunicação informal e atualizações rápidas. É importante que o WhatsApp, por ser um meio de comunicação mais difundido na sociedade, seja um espaço de utilização tanto pelos estudantes quanto pela organização. Uma boa forma de delimitar isto é por meio de um grupo de WhatsApp com todos os envolvidos no projeto, porém, delimitando que aquele espaço é exclusivamente para tratar de assuntos relacionados ao projeto, para evitar discussões paralelas sem relação com a extensão.
    \item Discord: utilizado para codificação conjunta, resolução de problemas, agendamento de reuniões, enquetes e videochamadas. É um aplicativo mais específico e de mais conhecimento dos estudantes. Pode ser mais interessante para uso interno da equipe, sem envolver a instituição parceira.
\end{itemize}