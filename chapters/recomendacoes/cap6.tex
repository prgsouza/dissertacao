\section*{\#R6 – Definir a Duração do Projeto com Metodologias Ágeis}

É imprescindível que o projeto possua as suas etapas bem definidas. A falta dessa definição pode gerar um tom de desorganização por parte do coordenador do projeto, o que pode causar ansiedade em alguns estudantes ou até mesmo desestimular os mesmos. 

Uma possível metodologia para projetos de extensão e inovação social é a metodologia Speedplay de \citeauthor{ferrario2014} (\citeyear{ferrario2014}), o qual é um método de desenvolvimento de inovação social por meio de artefatos digitais, em ambientes que exigem uma maior celeridade, além de grupos com difícil acesso, representando uma metodologia de execução rápida, que permite uma autonomia na forma organizacional para quem irá realizar a coordenação do projeto. Como possui o foco em artefatos digitais e prototipagem rápida e exige um ritmo acelerado, o Speedplay funciona bem com programação via Inteligência Artificial (Vibecoding), que será falado nos próximos capítulos, onde o Speedplay atua como bússola metodológica, enquanto o Vibecoding traduz essa bússola em código computacional. 

Um dos momentos importantes do Speedplay é o marco chamado por Ferrario de Ponto focal, onde será um momento de aceleração da colaboração e um importante momento de engajamento. As formas de execução desse ponto focal serão mais detalhadas nos próximos capítulos.

As fases do Speedplay são as seguintes:

\begin{enumerate}
    \item Preparar: \textit{ideathon} ou \textit{hackathon} como pontos focais, levantamento de requisitos, início de protótipos.
    \item Co-Criar: exploração das ideias e desenho de soluções.
    \item Construir: construção de MVP, testes e validação.
    \item Sustentar: consolidação do aprendizado e continuidade das soluções.
\end{enumerate}

Em projetos curtos, o Scrum pode ser adaptado para “mini-sprints”, exigindo encontros frequentes de \textit{Sprint Reviews} e testes de usuários.
