\section*{\#R6 – Utilizar Ferramentas de Gerenciamento de Tarefas}

O gerenciamento de tarefas é fundamental para o sucesso de projetos. Um bom planejamento e gerenciamento, além de reduzir a quantidade de erros que podem ocorrer no processo e na solução final, também evita retrabalho e permite uma otimização do tempo de todos os participantes.

Essas são algumas ferramentas básicas que podem auxiliar no gerenciamento das tarefas no projeto:

\begin{itemize}
    \item Jira: especializado em gestão de projetos de software, com rastreamento de bugs e relatórios.
    \item Notion: versátil, permite bases de dados, listas de tarefas e organização visual.
    \item Trello: simples e eficaz com quadros Kanban para backlog.
\end{itemize}

Apesar do uso dessas ferramentas, muitas vezes a comunicação informal acaba predominando como principal forma de gerenciamento. Essa não é a forma ideal, visto que não permite a documentação do projeto, o que pode dificultar para a posterior avaliação dos pontos críticos, o levantamento de lições aprendidas, e a consequente melhoria nas próximas execuções do projeto.