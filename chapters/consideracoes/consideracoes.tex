\chapter{Considerações Finais}
\label{chap:consideracoesfinais}

Como apontado ao longo do trabalho, a literatura acerca do tema abordado é muito escassa. Essa trabalho visou preencher uma lacuna existente na literatura científica, sobre pesquisas relacionadas a Inovação Social Aberta no Terceiro Setor por meio da Extensão Universitária, além de observar e mapear quais são essas atividades, e quais são os principais desafios e sucessos.

Através das observações da literatura e dos projetos de extensão CInbora Impactar e CInovação Social já citados ao longo do texto, foi construído um Guia de Práticas em Inovação Social Aberta na Extensão Universitária, como produto técnico exigido pelo Mestrado Profissional em Ciência da Computação, baseado nas recomendações demonstradas por esta dissertação, que pode servir como referência para docentes, discentes, coordenadores de extensão e instituições do terceiro setor que possuam interesse em praticar a Inovação Social Aberta no seu contexto, através da extensão universitária.

Essa pesquisa buscou aproximar a teoria e a prática, além de contribuir com a consolidação da inovação social aberta, principalmente no meio universitário e obteve bons resultados com projetos contando com um grande engajamento por parte dos estudantes e das instituições envolvidas, além na grande qualidade dos artefatos desenvolvidos.

Para trabalhos futuros, além do fortalecimento do guia, da validação com outros projetos e de suas melhorias, também existe a possibilidade da realização de plataformas digitais que potencializem esse guia, e demonstrem de forma mais prática como \textit{toolkits}, onde os interessados poderão navegar pelo guia, ver vídeo-aulas ensinando a realizarem as práticas e as metodologias propostas, além da proposição de cursos para a utilização desta plataforma e também da disseminação da prática da Inovação Social Aberta através da Extensão Universitária, para além dos cursos do Centro de Informática, como foram observados neste projeto. 

Também podem ser realizados trabalhos de fortalecimento da curricularização da extensão universitária por meio da criação de guias e recomendações mais específicas para integração de projetos em mais de uma disciplina, através das mais variadas metodologias, como exemplo a Aprendizagem Baseada em Problemas ou Desafios.