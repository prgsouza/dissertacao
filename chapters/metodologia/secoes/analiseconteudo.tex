\section{ANÁLISE DE DADOS}
\label{analisededados}

A análise dos dados será realizado através dos seguintes passos:

\begin{itemize}
    \item \textbf{Transcrição}: Todas as entrevistas e grupos focais serão transcritos por meio de Inteligência Artificial, e serão posteriormente revisadas pelos pesquisadores, a fim de garantir fidelidade à fala dos participantes.
    \item \textbf{Organização dos dados:} Atribuição das falas a cada um dos entrevistados e posterior anonimização das falas, substituindo nomes por pseudônimos ou códigos e armazenamento seguro dos dados obtidos, e organização separada dos grupos focais e entrevistas. Em segunda execução, foi realizada análise de dados de entrevistas e grupos focais com os estudantes e colaboradores da ONG que foi aplicado o projeto de extensão. 
    \item \textbf{Análise de Conteúdo:} Será aplicada a Análise de Conteúdo de acordo com \citeauthor{bardin2011} (\citeyear{bardin2011}):
	\begin{itemize}
	    \item \textbf{Pré-análise:} Identificação das categorias consoante as falas dos participantes;
	\item \textbf{Exploração do material:} Codificação, classificação e agregação para agrupamento de falas com temas semelhantes para identificar possíveis padrões, por meio de ferramentas de análise;
	\item \textbf{Tratamento dos resultados:} inferência e interpretação dos dados obtidos e tratados.
	\end{itemize}


\end{itemize}