\chapter{Grupo focal com os estudantes}
\par\vspace{1\baselineskip}
\textbf{Metodologia}
\begin{itemize}
    \item Tipo de Pesquisa: Qualitativa
    \item Método de Coleta: Grupo focal
    \item População alvo: Estudantes participantes do projeto CInBora Impactar. Estudantes participantes do projeto CInovação Social.
\end{itemize}
\par\vspace{1\baselineskip}
\textbf{Roteiro de Grupo Focal}
\par\vspace{1\baselineskip}
\textbf{Informações básicas}
\begin{enumerate}
    \item Vamos iniciar conhecendo um pouco mais sobre vocês. Poderiam falar seus nomes e suas idades?  
    \item Vocês estagiam ou trabalham? Se sim, onde?  
    \item Qual a melhor forma de contatarmos vocês após essa entrevista, caso necessário?  
\par\vspace{1\baselineskip}
\textbf{Aspirações e valores}
    \item Quais as suas expectativas na sua graduação?  
    \item Quais as suas expectativas nesse projeto de extensão?  
\par\vspace{1\baselineskip}
\textbf{Experiência no projeto}
    \item Quais aspectos que vocês consideram mais positivos da condução do projeto por parte do professor?  
    \item Quais foram as dificuldades que vocês se depararam ao longo do projeto?  
    \item Foi necessário pivotar alguma etapa do projeto durante sua realização? Se sim, qual?  
    \item Como foi esse processo de mudança durante a execução do projeto?  
    \item Como tem sido a suas interações com os outros integrantes do seu grupo, com o professor e com os \textit{stakeholders}?
     - Para os participantes do CInbora Impactar: Da Prefeitura.
    - Para os participantes do CInovação Social: Da \gls{ONG}.
    \item Vocês acreditam que receberam o suporte necessário para a plena realização do projeto? Por quê?
\par\vspace{1\baselineskip}
\textbf{Sugestões de melhorias}
    \item Quais seriam as principais melhorias que vocês conseguem observar e pontuar na condução do projeto de extensão?  
    \item Como vocês acreditam que a comunicação entre os envolvidos do projeto poderia ser mais efetiva?  
    \item O que vocês mudariam na troca de saberes (interação dialógica) entre as instituições?  
    - Para os participantes do CInbora Impactar: Entre a Prefeitura e a Universidade.
    - Para os participantes do CInovação Social: Entre a \gls{ONG} e a Universidade.
\par\vspace{1\baselineskip}
\textbf{Informações complementares}
    \item Como vocês avaliariam o trabalho colaborativo ao longo da execução do projeto de extensão?
     - Para os participantes do CInbora Impactar: Entre a Prefeitura e a Universidade.
    - Para os participantes do CInovação Social: Entre a \gls{ONG} e a Universidade.
    \item O que vocês aprenderam dentro desta troca de saberes (interação dialógica) que considera mais relevante?  
     - Para os participantes do CInbora Impactar: Entre a Prefeitura e a Universidade.
    - Para os participantes do CInovação Social: Entre a \gls{ONG} e a Universidade.
    \item Vocês acreditam que a abordagem de conciliação entre a disciplina e projeto de extensão seria adequada para outras disciplinas? Por quê? 
    \item Quais \textit{soft-skills} vocês acreditam que tiveram a oportunidade de desenvolver com este projeto?
    \item Como esse projeto influenciou sua percepção sobre inovação social e extensão universitária? 
    \item Gostariam de compartilhar alguma outra informação que não foi perguntada ao longo do grupo?  
\end{enumerate}
\par\vspace{1\baselineskip}
