\chapter{SÍNTESE DO TERMO DE CONSENTIMENTO LIVRE E ESCLARECIDO}

Convidamos o (a) Sr. (a) para participar como voluntário (a) da pesquisa Inovação Social Aberta no Terceiro Setor por meio da Extensão Universitária Todas as suas dúvidas podem ser esclarecidas com o responsável por esta pesquisa. O (a) senhor (a) estará livre para decidir participar ou recusar-se. Caso não aceite participar, não haverá nenhum problema, desistir é um direito seu, bem como será possível retirar o consentimento em qualquer fase da pesquisa, também sem nenhuma penalidade.

\textbf{Estratégia de Coleta de Dados}
\begin{itemize}
    \item Observação participante do pesquisador durante a execução do projeto e nos encontros entre os estudantes e a Prefeitura;
    \item Grupos focais com estudantes participantes do projeto CInbora Impactar e CInovação Social;
    \item Entrevistas semiestruturadas com funcionários da prefeitura participantes do projeto CInbora Impactar;
    \item Entrevistas semiestruturadas com colaboradores da \gls{ONG} Gris Social, participantes do projeto CInovação Social;
    \item Formulários online para os estudantes participantes do projeto impossibilitados e/ou não interessados em participar do grupo focal.
\end{itemize}

\textbf{Aspectos Éticos}
\begin{enumerate}
    \item \textbf{Confidencialidade}: Todas as informações que forem dadas por você serão totalmente confidenciais e suas respostas não serão associadas ao seu nome e/ou qualquer outro dado que possibilite a sua identificação, e quaisquer citações diretas serão totalmente anônimas.
    \item \textbf{Voluntariedade}: A participação nesta pesquisa não é obrigatória, e você tem o pleno direito de se recusar a responder qualquer pergunta ou cancelar sua participação durante qualquer momento desta pesquisa, sem quaisquer tipos de penalização ou consequência.
    \item \textbf{Propósito da pesquisa}: O propósito é compreender a percepção dos \textit{stakeholders} do projeto sobre a interação com os membros do projeto, possíveis melhorias de comunicação, a qualidade das soluções tecnológicas realizadas e o alinhamento de expectativas ao longo da execução do projeto.
    \item \textbf{Riscos e Benefícios}:
    \begin{itemize}
        \item Ansiedade: Pode ocorrer de alguns participantes se sentirem ansiosos ao serem questionados sobre seus desempenhos e interações ao longo do projeto de extensão, em decorrência de acreditarem que isso pode influenciar na avaliação por parte do Professor (no caso dos estudantes participantes do projeto de extensão), ou por parte de algum superior  (no caso dos colaboradores)
        \item Fadiga cognitiva: Uma duração mais longa pode demandar um grande esforço mental e assim causar algum tipo de estresse ou desconforto no participante
        \item Vazamento de dados: Mesmo com a garantia de anonimato para os entrevistados, a gravação e o vazamento de dados pode representar um risco para esse anonimato, expondo assim os participantes.
    \end{itemize}
    \textbf{Mitigação dos riscos:}
    \begin{itemize}
        \item Ansiedade: Criar um ambiente acolhedor e seguro, além de informar os recursos psicológicos disponíveis na Universidade.
        \item Fadiga cognitiva: Reforçar a voluntariedade da participação, e que os mesmos podem sair da entrevista ou do grupo focal a qualquer momento, e também oferecer pausas durante a entrevista e grupo focal.
        \item Pressão social: Moderar o grupo focal ativamente, garantindo um espaço de participação para todos os participantes, e caso necessário, dividir em subgrupos menores. Informar aos participantes que as respostas divergentes e individuais são importantes para a pesquisa.
        \item Vazamento de dados: Utilizar códigos nas transcrições e na análise de dados, e informar aos participantes que os dados serão armazenados em ambiente seguro.
    \end{itemize}
    \textbf{Benefícios:} Apesar de não oferecer nenhum benefício direto e imediato, sua participação irá auxiliar na melhoria do projeto de extensão em execuções futuras.
    \item \textbf{Duração}: A entrevista durará entre 30 e 45 minutos.
\end{enumerate}

Esclarecemos que os participantes dessa pesquisa têm plena liberdade de se recusar a participar do estudo e que esta decisão não acarretará penalização por parte dos pesquisadores. Todas as informações desta pesquisa serão confidenciais e serão divulgadas somente em eventos ou publicações científicas, não havendo identificação dos voluntários, a não ser entre os responsáveis pelo estudo, sendo assegurado o sigilo sobre a sua participação. Os dados coletados nesta pesquisa (gravações e transcrições) ficarão armazenados em disco rígido de computador pessoal, sob a responsabilidade do pesquisador pelo período de mínimo 5 anos após o término da pesquisa. Nada lhe será pago e nem será cobrado para participar desta pesquisa, pois a aceitação é voluntária. Em caso de dúvidas relacionadas aos aspectos éticos deste estudo, o (a) senhor (a) poderá consultar o Comitê de Ética em Pesquisa da \gls{UFPE}.

