\chapter{TERMO DE CONSENTIMENTO LIVRE E ESCLARECIDO}

Convidamos o (a) Sr. (a) para participar como voluntário (a) da pesquisa Inovação Social Aberta no Terceiro Setor por meio da Extensão Universitária, que está sob a responsabilidade do (a) pesquisador (a) Pedro Rodolfo Gomes de Souza.

Também participam desta pesquisa a pesquisadora: Roberta Baudel Francisco. e está sob a orientação de: Kiev Santos da Gama.

Todas as suas dúvidas podem ser esclarecidas com o responsável por esta pesquisa. Apenas quando todos os esclarecimentos forem dados e você concorde com a realização do estudo, pedimos que rubrique as folhas e assine ao final deste documento, que está em duas vias. Uma via lhe será entregue e a outra ficará com o pesquisador responsável.

O (a) senhor (a) estará livre para decidir participar ou recusar-se. Caso não aceite participar, não haverá nenhum problema, desistir é um direito seu, bem como será possível retirar o consentimento em qualquer fase da pesquisa, também sem nenhuma penalidade.

\textbf{Contexto}

Esta investigação é uma iniciativa do Estudante de Mestrado do Programa de Pós-Graduação Profissional em Ciência da Computação do CIn/UFPE Pedro Rodolfo Gomes de Souza, orientado pelo Professor Kiev Santos da Gama da mesma instituição. A motivação é estritamente acadêmica e visa a compreensão da percepção dos \textit{stakeholders} envolvidos, avaliando a interação com os membros do projeto, possíveis melhorias de comunicação, a qualidade das soluções tecnológicas realizadas e o alinhamento de expectativas ao longo da execução do projeto. Os resultados podem ser utilizados em publicações científicas, respeitando a confidencialidade e o anonimato de cada um de seus participantes. Caso tenha alguma dúvida ou necessite de mais informações, favor entrar em contato com o estudante Pedro.

\textbf{Objetivos Gerais}

Estudar práticas que viabilizam a Inovação Social Aberta no terceiro setor através da Extensão Universitária, nos projeto CInbora Impactar em parceria com a Prefeitura do Recife e CInovação Social em parceria com a \gls{ONG} Gris Social, promovendo a colaboração entre a Universidade, Organizações do Terceiro Setor e outros \textit{stakeholders}.

\textbf{Objetivos específicos}
\begin{itemize}
    \item Avaliar os impactos que a Inovação Social Aberta por meio da Extensão Universitária pode proporcionar as instituições do terceiro setor.
    \item Mapear os atores envolvidos na prática de Inovação Social Aberta no projeto de extensão CInbora Impactar em parceria com a Prefeitura do Recife e no projeto de extensão CInovação Social em parceria com a \gls{ONG} Gris Social.
    \item Observar os principais desafios vivenciados, a experiência dos atores envolvidos e realizar uma análise destes dados;
    \item Desenvolver um plano de pesquisa-ação visando atuar numa próxima execução deste projeto de extensão;
\end{itemize}

\textbf{Problemática da Pesquisa}

Este trabalho visa tratar a seguinte pergunta: Como viabilizar a prática da Inovação Social Aberta no Terceiro Setor por meio da Extensão Universitária?


\textbf{Estratégia de Coleta de Dados}
\begin{itemize}
    \item Observação participante do pesquisador durante a execução do projeto e nos encontros entre os estudantes e a Prefeitura;
    \item Grupos focais com estudantes participantes do projeto CInbora Impactar e CInovação Social;
    \item Entrevistas semiestruturadas com funcionários da prefeitura participantes do projeto CInbora Impactar;
    \item Entrevistas semiestruturadas com colaboradores da \gls{ONG} Gris Social, participantes do projeto CInovação Social;
    \item Formulários online para os estudantes participantes do projeto impossibilitados e/ou não interessados em participar do grupo focal.
\end{itemize}

\textbf{Justificativa}


Na literatura, ainda existem poucos artigos que tratam da inovação social e desse diálogo e interação entre a academia e a sociedade, sendo assim, é necessário estudar como ocorre a Inovação Social Aberta nas universidades, se existem práticas através da Extensão Universitária, compreender as nuances e desafios e propor posteriormente o início da implantação de processos que possam potencializar esse processo para ambos atores. \cite{klaumann2023}.

Visando auxiliar as ONGs no atendimento de algumas de suas necessidades e finalidades, unindo a necessidade de atender as normativas já citadas de Curricularização da Extensão, e considerando o cenário desafiador das práticas extensionistas e das complexidades enfrentadas pelas ONGs, a Inovação Social Aberta pode ser o elo que irá fortalecer ambas as partes envolvidas, a Universidade e a Sociedade, sendo um catalisador e propulsor destas.

\textbf{Aspectos Éticos}

\begin{enumerate}
    \item \textbf{Confidencialidade}: Todas as informações que forem dadas por você serão totalmente confidenciais e suas respostas não serão associadas ao seu nome e/ou qualquer outro dado que possibilite a sua identificação, e quaisquer citações diretas serão totalmente anônimas.
    \item \textbf{Voluntariedade}: A participação nesta pesquisa não é obrigatória, e você tem o pleno direito de se recusar a responder qualquer pergunta ou cancelar sua participação durante qualquer momento desta pesquisa, sem quaisquer tipos de penalização ou consequência.
    \item \textbf{Propósito da pesquisa}: O propósito é compreender a percepção dos \textit{stakeholders} do projeto sobre a interação com os membros do projeto, possíveis melhorias de comunicação, a qualidade das soluções tecnológicas realizadas e o alinhamento de expectativas ao longo da execução do projeto.
    \item \textbf{Riscos e Benefícios}:
    \begin{itemize}
        \item Ansiedade: Pode ocorrer de alguns participantes se sentirem ansiosos ao serem questionados sobre seus desempenhos e interações ao longo do projeto de extensão, em decorrência de acreditarem que isso pode influenciar na avaliação por parte do Professor (no caso dos estudantes participantes do projeto de extensão), ou por parte de algum superior  (no caso dos colaboradores)
        \item Fadiga cognitiva: Uma duração mais longa pode demandar um grande esforço mental e assim causar algum tipo de estresse ou desconforto no participante
        \item Vazamento de dados: Mesmo com a garantia de anonimato para os entrevistados, a gravação e o vazamento de dados pode representar um risco para esse anonimato, expondo assim os participantes.
    \end{itemize}
    \textbf{Mitigação dos riscos:}
    \begin{itemize}
        \item Ansiedade: Criar um ambiente acolhedor e seguro, além de informar os recursos psicológicos disponíveis na Universidade.
        \item Fadiga cognitiva: Reforçar a voluntariedade da participação, e que os mesmos podem sair da entrevista ou do grupo focal a qualquer momento, e também oferecer pausas durante a entrevista e grupo focal.
        \item Pressão social: Moderar o grupo focal ativamente, garantindo um espaço de participação para todos os participantes, e caso necessário, dividir em subgrupos menores. Informar aos participantes que as respostas divergentes e individuais são importantes para a pesquisa.
        \item Vazamento de dados: Utilizar códigos nas transcrições e na análise de dados, e informar aos participantes que os dados serão armazenados em ambiente seguro.
    \end{itemize}
    \textbf{Benefícios:} Apesar de não oferecer nenhum benefício direto e imediato, sua participação irá auxiliar na melhoria do projeto de extensão em execuções futuras.
    \item \textbf{Duração}: A entrevista durará entre 30 e 45 minutos.
\end{enumerate}

Esclarecemos que os participantes dessa pesquisa têm plena liberdade de se recusar a participar do estudo e que esta decisão não acarretará penalização por parte dos pesquisadores. Todas as informações desta pesquisa serão confidenciais e serão divulgadas somente em eventos ou publicações científicas, não havendo identificação dos voluntários, a não ser entre os responsáveis pelo estudo, sendo assegurado o sigilo sobre a sua participação. Os dados coletados nesta pesquisa (gravações e transcrições) ficarão armazenados em disco rígido de computador pessoal, sob a responsabilidade do pesquisador Pedro Rodolfo Gomes de Souza, pelo período de mínimo 5 anos após o término da pesquisa.

Nada lhe será pago e nem será cobrado para participar desta pesquisa, pois a aceitação é voluntária, mas fica também garantida a indenização em casos de danos, comprovadamente decorrentes da participação na pesquisa, conforme decisão judicial ou extrajudicial. Se houver necessidade, as despesas para a sua participação serão assumidas pelos pesquisadores (ressarcimento de transporte e alimentação).

Em caso de dúvidas relacionadas aos aspectos éticos deste estudo, o (a) senhor (a) poderá consultar o Comitê de Ética em Pesquisa Envolvendo Seres Humanos da UFPE no endereço: \textbf{Avenida da Engenharia s/n — 1º Andar, sala 4 - Cidade Universitária, Recife–PE, CEP: 50740-600, Tel.: 81 2126.8588 — e-mail: cephumanos.ufpe@ufpe.br}
\par\vspace{1\baselineskip}

\textbf{CONSENTIMENTO DA PARTICIPAÇÃO DA PESSOA COMO VOLUNTÁRIO (A)}
\par\vspace{1\baselineskip}

Eu, NOME DO PARTICIPANTE, CPF, abaixo assinado, após a leitura (ou a escuta da leitura) deste documento e de ter tido a oportunidade de conversar e ter esclarecido as minhas dúvidas com o pesquisador responsável, concordo em participar do estudo Inovação Social Aberta no Terceiro Setor por meio da Extensão Universitária, como voluntário (a). Fui devidamente informado (a) e esclarecido (a) pelo(a) pesquisador (a) sobre a pesquisa, os procedimentos nela envolvidos, assim como os possíveis riscos e benefícios decorrentes de minha participação. Foi-me garantido que posso retirar o meu consentimento a qualquer momento, sem isto levar a qualquer penalidade.

Local e data 

Assinatura do participante: 
\par\vspace{1\baselineskip}

Presenciamos a solicitação de consentimento, esclarecimentos sobre a pesquisa e o aceite do voluntário em participar. (02 testemunhas não ligadas à equipe de pesquisadores):

Assinatura das testemunhas