\chapter{Entrevista com os \textit{stakeholders} da Prefeitura do Recife e ONG Gris Social} 
\textbf{Roteiro da Entrevista Semi-Estruturada}
\par\vspace{0.5\baselineskip}
\textbf{Informações básicas}
\begin{enumerate}
    \item Vamos iniciar conhecendo um pouco mais sobre você. Qual seu nome?
    \item Quantos anos você tem?
    \item Qual sua área de formação?
    \item Qual o cargo que você ocupa atualmente na Prefeitura do Recife, e em qual órgão? - para funcionários da Prefeitura
    \item Qual é a função que você exerce atualmente? - para colaboradores da \gls{ONG}
    \item Qual a melhor forma de contatarmos você após essa entrevista, caso necessário?
\par\vspace{0.5\baselineskip}
\textbf{Experiência prévia}
    \item Você já participou de algum outro projeto e/ou iniciativa envolvendo organizações não governamentais e/ou organizações do terceiro setor? Se sim, poderia especificar quais? - para funcionários da Prefeitura
    \item Você já possui experiência prévia com concepção, desenvolvimento e/ou execução de atividades/práticas inovativas? Se sim, poderia especificar mais?
    \item Você já participou de alguma atividade de interação? Caso sim, como você avalia essa troca de saberes?
    - Para os funcionários da Prefeitura: Entre a Universidade e a Prefeitura?
    - Para os colaboradores da \gls{ONG}: Entre a Universidade e a ONG?
\par\vspace{0.5\baselineskip}
\textbf{Experiência no projeto}
    \item Como você descreveria sua participação no desenvolvimento do projeto até este momento?
    \item Quais aspectos que você considerou mais positivos da condução do projeto, através da parceria com a Universidade?
    \item Quais foram as dificuldades que você se deparou ao longo do projeto, através da parceria com a Universidade?
    \item Como tem sido a sua interação com o professor e com os estudantes?
    \item Você acredita que as entregas e demandas, além do acompanhamento destas, foram bem definidas por parte do professor? Por quê?
\par\vspace{0.5\baselineskip}
\textbf{Avaliação das soluções tecnológicas realizadas}
    \item Como você avaliaria a qualidade das soluções tecnológicas realizadas pelos estudantes até o momento?
    \item Como as soluções tecnológicas realizadas pelos estudantes atenderam (ou não) as expectativas?
     - Para os funcionários da Prefeitura: da Prefeitura e das \gls{ONG}s?
    - Para os colaboradores da \gls{ONG}: da \gls{ONG}?
\par\vspace{0.5\baselineskip}
\textbf{Mudanças e melhorias}
    \item Houve a necessidade de mudanças ou ajustes significativos ao longo do projeto?
    \item Essas mudanças propostas, na sua opinião, colaboraram positiva ou negativamente com a realização do projeto? Você atribuiria isso a algo em específico?
    \item O que você mudaria na troca de saberes (interação dialógica) entre a instituição que você participa e Universidade (professor e estudantes)?
     - Para os funcionários da Prefeitura: Entre a Prefeitura e a Universidade.
    - Para os colaboradores da \gls{ONG}: Entre a \gls{ONG} e a Universidade.
\par\vspace{0.5\baselineskip}
\textbf{Informações complementares}
    \item O que você aprendeu dentro desta troca de saberes (interação dialógica) entre Universidade e a instituição que você participa que considera mais relevante?
     - Para os funcionários da Prefeitura: Entre a Prefeitura e a Universidade.
    - Para os colaboradores da \gls{ONG}: Entre a \gls{ONG} e a Universidade.
    \item Gostaria de compartilhar alguma outra informação que não foi perguntada ao longo da entrevista?
\end{enumerate}
