% resumo em português
\begin{resumo}[Resumo] 
A Inovação Social segundo a Organização para a Cooperação e Desenvolvimento Econômico (OCDE) em sua Recomendação do Conselho sobre a Economia Social e
Solidária e a Inovação Social \cite{ocde2024social} é a área que busca soluções possíveis para as questões sociais, visando a melhoria do bem-estar e qualidade de vida das pessoas e comunidades em geral, através de serviços, produtos e outros. 

Um dos paradigmas existentes para a prática inovativa é a Inovação Aberta, descrita como o uso de conhecimento externo as organizações (ou exportação do conhecimento interno) visando a potencialização dos processos inovativos, num processo benéfico para ambas as organizações envolvidas. 

A interseção entre as duas áreas é nomeada de Inovação Social Aberta, em decorrência de, segundo \citeauthor{chesbrough2014} (\citeyear{chesbrough2014}), a Inovação Aberta muitas vezes está focada estritamente no setor privado. 

Uma das formas de execução dessa Inovação Social Aberta é por meio da Extensão Universitária, que tem por objetivo, segundo o Conselho Nacional de Educação do Ministério da Educação (2018), a “interação transformadora entre as instituições de ensino superior e os outros setores da sociedade, por meio da produção e da aplicação do conhecimento, em articulação permanente com o ensino e a pesquisa.”. 

Em função disto, este trabalho visa contribuir na prática da Inovação Social Aberta no terceiro setor promovidas pelas Universidades, por meio de um Guia de Práticas, analisadas e vivenciadas em projetos de Extensão já executados, como o Bora Impactar e o CInovação Social, voltada para o desenvolvimento de artefatos digitais como veículos de Mudança Social, como preconizado por \citeauthor{ferrario2014}, (\citeyear{ferrario2014}).
% \noindent %- o resumo deve ter apenas 1 parágrafo e sem recuo de texto na primeira linha, essa tag remove o recuo. Não pode haver quebra de linha.

 \vspace{\onelineskip}
    
 \noindent
 \textbf{Palavras-chaves}: Inovação social aberta; extensão universitária; terceiro setor.
\end{resumo}



% resumo em inglês
\begin{resumo}[Abstract]
\begin{otherlanguage*}{english}

 %\noindent
Social Innovation, according to the Organisation for Economic Co-operation and Development (OECD) in its Council Recommendation on the Social and Solidarity Economy and Social Innovation \cite{ocde2024social}, is the field that seeks feasible solutions to social issues, aiming at the improvement of well-being and quality of life of individuals and communities in general, through services, products, and other means.

One of the existing paradigms for innovative practice is Open Innovation, described as the use of external knowledge by organizations (or the outward transfer of internal knowledge) with the purpose of enhancing innovation processes, in a mutually beneficial exchange between the organizations involved.

The intersection of these two fields is referred to as Open Social Innovation, since, according to \citeauthor{chesbrough2014} (\citeyear{chesbrough2014}), Open Innovation is often strictly focused on the private sector.

One of the ways in which Open Social Innovation can be implemented is through University Extension programs, which, according to the National Education Council of the Ministry of Education (2018), aim at “the transformative interaction between higher education institutions and other sectors of society, through the production and application of knowledge, in permanent articulation with teaching and research.”

In this context, this study seeks to contribute to the practice of Open Social Innovation in the third sector promoted by universities, through a Practice Guide, analyzed and experienced in Extension projects already carried out, such as Bora Impactar and CInovação Social. These initiatives are oriented toward the development of digital artifacts as vehicles for Social Change, as advocated by \citeauthor{ferrario2014} (\citeyear{ferrario2014}).



   \vspace{\onelineskip} 
 
   \noindent 
   \textbf{Keywords}: Open Social Innovation; university outreach; third sector.
 \end{otherlanguage*}
 \end{resumo}
